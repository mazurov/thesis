Il trigger \`e un elemenco cruciale negli esperimenti di fisica ai collisori adronici. 
In questa tesi viene sviluppato uno strumento di profilazione del software, che \`e 
utile per analizzare e migliorare le prestazioni del trigger di alto livello dell'esperimento 
\lhcb. Questo strumento \`e in grado di identificare i punti in cui i tempi di esecuzione 
degli algoritmi del trigger sono lenti, permettendo quindi l'ottimizzazione della 
velocit\`a di decisione del trigger e minimizzandone i tempi morti durante la presa dati 
dell'esperimento. 

Il trigger di \lhcb \`e molto efficiente e permette di studiare con precisione i decadimenti 
di particelle contenenti quark pesanti con muoni nello stato finale. 
In questa tesi \`e stato effettuato uno studio della produzione di mesoni \chib a \lhcb, 
effettuato sui dati raccolti in collisioni protone-protone a energie di 7 e 8 TeV nel 
centro di massa. La luminosit\`a integrata corrispondente \`e di 3\invfb. 
Si ricostruiscono i decadimenti radiativi della \chib in  
\Y1S, \Y2S e \Y3S, in cui le $\Upsilon$ decadono in due muoni. 
Le frazioni di mesoni $\Upsilon$ che originano da decadimenti di particelle \chib 
vengono misurate in funzione dell'impulso trasverso della 
$\Upsilon$ nell'intervallo di rapidit\`a 
$2.0 < y^{\Upsilon} < 4.5$. Gli intervalli di impulso trasverso analizzati per 
i decadimenti in \Y1S, \Y2S e \Y3S sono rispettivamente  $6<p_T^{\Y1S}<40\gevc$, 
$18<p_T^{\Y2S}<40\gevc$ e $27<p_T^{\Y3S}<40\gevc$. 
Viene inoltre effettuata la prima misura della frazione di \Y3S prodotte in decadimenti radiativi della 
\chibThreeP, ottenendo come risultato rispettivamente 
(42 $\pm$ 12\stat${}^{+8.9}_{-11.6}\syst^{+2.7}_{-3.1}\systpol) \%$ e 
(41 $\pm$ 8\stat${}^{+1.3}_{-8.6}\syst^{+2.6}_{-3.1}\systpol) \%$ a \sqs=7 e 8 \tev. 
La misura della massa del mesone \chiboneThreeP d\`a come risultato $10{,}508 \pm 2\stat \pm
8\syst \mevcc$.


