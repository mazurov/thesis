\chapter{The LHCb experiment}
\label{ch_lhcb}

The LHCb is a dedicated heavy flavor physics experiment situated at the LHC
collider.
The primary purpose of this experiment is searching for new physics in CP
violation and the rare decays of hadrons containing beauty and charm quarks.

This chapter gives a brief overview of the LHCb detector, describes its
sub-detectors and their performances. More detailed information and references
on LHCb design and operation can be found in \cite{Alves:2008zz}.

Firstly the properties of the LHC accelerator are presented, followed by an
overview of the LHCb detector. Then the outline of sub-detectors used for
tracking and particle identification is given, followed by the description of
trigger system that is an important part for selecting the most interesting
events while reducing the event rate. Finally, the software used at the LHCb
is described.

\section{The LHC}
\label{ch_lhcb:lhc}

The Large Hadron Collider (LHC) is a circular proton-proton collider  located
at the European Organization for Nuclear Research (CERN), on the French-Swiss
border, near Geneva. Before the injection of the proton bunches into the main
LHC ring protons pass through series of low-energy pre-accelerators, as shown
in Fig.~\ref{fig:lhc}.

\begin{figure}[tb]
\centering
\includegraphics[width=300px]{figs/lhc.png}
\caption{\small The LHC Accelerator System}
\label{fig:lhc}
\end{figure}

The initial linear accelerator (LINAC2) accelerates protons energy to 50 MeV,
then they are fed through the Proton Synchrotron Booster (BOOSTER) which
accelerates them to 26 GeV, and finally protons are injected  into the LHC
complex at an energy of 450 GeV.

The four main LHC experiments situated at the beam crossing points shown in
figure~\ref{fig:lhc}: ATLAS, ALICE, CMS, LHCb. ALICE dedicated to heavy ion
physics. ATLAS and CMS are general purpose detectors, which primary goal
is to discover production of new particles. More details on the 
LHCb experiment, data from which is studied in this thesis, are given
in the next section.

The new particles are expected to have large masses and their production
processes have small cross sections, so the LHC machine is designed with both
a center-of-mass energy and a luminosity as large as possible.

The operation of the LHC can be shown as follows: two bunch of protons move in
opposite direction in orbit around 27 km circumference of the accelerator by
the magnetic field of superconducting magnets. A temperature 2 K is preserved
for magnets' coils to generate a maximum magnetic field of 8 T.  This field
allows to produce the design center-of-mass energy of $\sqrt{s}=14 TeV$.
Finally the bunches are designed to collide with a frequency of 40 MHz at the
interaction points to archive a design luminosity of $10^{34}cm^{-2}c^{-1}$.

The main LHC design parameters are shown in Table~\ref{tab:lhc}.

\begin{table}[t]
    \caption{\small The main LHC design parameters}
    \centering
    \begin{tabular}{r|r}
        Circumference  &  27 km\\
        Center-of-mass energy &  14 TeV\\
        Injection energy  &  450 GeV\\
        Field at 2 $\times$ 450 GeV  &  0.535 T\\
        Field at 2 $\times$ 7 Tev & 8 T\\
        Helium temperature & 2 K\\
        Luminosity & $10^{34}cm^{-2}s^{-1}$\\
        Bunch spacing  & 25 ns\\
        Luminosity lifetime & 10 h\\
        Time between 2 fills  &  7 h\\
    \end{tabular}
    \label{tab:lhc}
\end{table}

\section{The LHCb}
\label{ch_lhcb:lhcb}

The \lhcb is an experiment dedicated to precision measurements of CP violation
and rare decays of hadrons containing  beauty(b) and charmed(c) quarks as
an indirect search of new physics beyond the Standard Model.

The \lhcb detector is a forward single-arm spectrometer with forward angular
momentum coverage from 10 mrad to 300 mrad in the bending plane and 10 mrad to
250 mrad in the non-bending plane. The choice of the unique LHCb geometry is
justified by the fact, that b-hadrons are predominantly produced in narrow
angular cones in the same forward and backward directions.

\begin{figure}[tb]
\begin{center}
\includegraphics[width=300px]{lhcb.png}
\end{center}
\caption{\small Schematic layout of the LHCb detector \cite{Alves:2008zz}. 
The interaction point where the protons collide is on the left of the figure, 
and sub-detectors are labeled.}
\label{fig:lhcb}
\end{figure}

LHCb allows the full reconstruction of exclusive decays of the b- and c-hadrons
in a variety of leptonic, semi-leptonic and purely hadronic final states. 
In order to achieve this goal and extract the physics of interest, 
specialized sub-detectors involved within the LHCb detector to perform the 
following major tasks:

\begin{itemize}

\item {\bf Precision vertexing}: a sufficient separation between primary and 
secondary vertices is required to efficiently select b-hadron candidates 
and allow time dependent analyses to be performed. Such measurements are 
performed by the VErtex LOcator (VELO).

\item {\bf Invariant mass determination}: an finest invariant mass
resolution is required in order to maximize the significance of signal
with respect  to background. Therefore, precision energy and momentum
estimates of reconstructed tracks must be performed. This is achieved by LHCb’s 
tracking  and calorimetry  
systems.

\item {\bf Particle identification}: hadronic decays of b- and c-hadrons, 
having identical topology but different flavour content in the final state, 
may peak at a common invariant mass; additional information is required to 
distinguish them from one another. 
Discrimination between charged hadrons (particularly pions and kaons) is 
achieved with a high performance Ring Imaging CHerenkov (RICH) system, 
whilst electrons, photons and muons are identified via the 
Calorimeter and Muon systems, respectively.

\item {\bf Flexible and robust trigger and data acquisition}: this is required 
in order to cope with rapid changes in running conditions and physics 
interests. A dedicated multi-stage trigger, capable of selecting many different 
final states in an hadronic environment, reduces the data rate from the initial 
\~ 40~MHz of "visible interactions" to \~ 5~kHz which is suitable for 
offline storage and analysis.
\end{itemize}

Fig.~\ref{fig:lhcb} presents the layout of the detector sub-systems 
within the LHCb detector. More details on each sub-detector will be given in 
the next sections.

\section{Tracking system}

The tracking system is an important part of the LHCb detector that collects such
information about charged particles as vertexing (determining the distance
between the production and the decay vertex of the $\chi_b$) and
momentum reconstruction. These two together used for reconstruction of the mass,
the angular and the proper time resolution,  that are important for signal
selection and background suppression during the offline analysis of
$\chi_b \rightarrow \Upsilon \gamma$. Besides this, momentum and decay distance
information about momentum and decay distance are used in the trigger.

The LHCb tracking system is composed of a dipole magnet, the VELO and four 
planar tracking systems: the Tracker Turicensis (TT) upstream of the dipole 
magnet and three tracking stations T1, T2 and T3 downstream of the magnet. 
The latter three stations cover the entire geometrical acceptance of the 
spectrometer. To achieve the excellent tracking performance and also due to 
track multiplicity considerations, these three stations are composed of two 
distinct parts called the Inner Tracker (IT) and Outer Tracker (OT). 
The VELO, TT and IT use silicon strip technology while straw tubes are 
employed in the OT. In fact, the TT and IT share a common technology, 
called collectively the Silicon Tracker (ST). They have a very similar layout
sharing the same silicon microstrip technology with a strip pitch
of \~ 200$\mum$. Each of the four ST stations is composed of four detector 
layers with the strip directions arranged in a so called x-u-v-x layout: 
the first and fourth layers have vertical readout strips, while second (u) 
and third (v) layers have the strips rotated by a stereo angle of $+5^\circ$ 
and $-5^\circ$ respectively. This layout is designed to have the best hit
resolution in the x direction (in the bending plane), without losing the 
stereo measurement of the tracks.

\subsection{Vertex Locater}

To provide precise measurements of track coordinates close to the interaction 
region, the Vertex Locator (VELO), consisting of a series of silicon modules, 
is arranged along the beam direction. It is used to identify the detached 
secondary vertices typical for b-hadron decays and makes it possible to meet 
the requirement to reconstruct  $\chi_b \rightarrow \Upsilon \gamma$ decays
with a proper time resolution good  enough to resolve the fast time-dependent
oscillations of the CP asymmetry.

To provide accurate measurements of the position of the vertices, the silicon
modules of the VELO are placed closed to the beam axis, namely at 8 mm.
In order to detect the majority of the tracks originating
from the beam spot ($\sigma$ = 5.3 cm), the detector is designed such that
tracks emerging up to z = 10.6 cm downstream from the nominal interaction
point cross at least 3 VELO stations, for a polar angular window between 15
and 300 mrad, as shown in Fig.~\ref{fig:velooverview}.

\begin{figure}[tb]
\begin{center}
\includegraphics[width=300px]{velooverview.png}
\end{center}
\caption{\small The setup of the VELO silicon modules along the beam direction.
The left two pairs form the pile-up system. Indicated are the average crossing
angle for minimum bias events (60 mrad), and the minimal (15 mrad) and
maximal (390 mrad) angle for which at least 3 VELO stations are crossed.
390 mrad is the opening angle of a circle that encloses a rectangular opening
angle of 250 x 300 mrad}
\label{fig:velooverview}
\end{figure}

To enable fast reconstruction of tracks and vertices in the LHCb trigger,
a cylindrical geometry with silicon strips measuring $r\phi$ coordinates is
chosen for the modules.

The strips of the r sensor are concentric semi-circles, the $\phi$ sensors 
measure a coordinate almost orthogonal to the r-sensor. The geometry is shown in
Fig.~\ref{fig:velomodule}. A 2D reconstruction in the r-z plane alone allows to
detect tracks originating from close to the beam line in the high-level trigger.
These measurements are used to compute the impact parameter of tracks with
respect to the production vertex, which is used in the trigger to discriminate 
between signal and background. The level-0 trigger uses information from the
pile-up veto system, two stations located upstream, which make it possible to
reject events with multiple pp interactions in one beam crossing.

\begin{figure}
        \centering
        \begin{subfigure}[b]{0.5\textwidth}
            \centering
            \includegraphics[width=\textwidth]{velosensor}
            \caption{\velo $\phi$-sensor}
            \label{fig:velosensor}
        \end{subfigure}%
        ~ %add desired spacing between images, e. g. ~, \quad, \qquad etc.
          %(or a blank line to force the subfigure onto a new line)
        \begin{subfigure}[b]{0.5\textwidth}
            \centering
            \includegraphics[width=\textwidth]{velomodule}
            \caption{A VELO module and its dimensions.}
            \label{fig:velomodule}
        \end{subfigure}
        \caption{The \velo r- and $\phi$- sensors.}\label{fig:veloschema}
\end{figure}



The setup of the \velo is as follows. The half disc sensors are arranged in pairs of
r and $\phi$ sensors and are mounted back-to-back. The sensors are 300 $\mu m$
thick, radiation tolerant, n-implants in n-bulk technology [35]. The minimal
pitch of both the r and the $\phi$ sensors is 32 $\mum$, linearly increasing
towards the outer radius at 41.9 mm. To reduce the strip occupancy and pitch at
the outer edge of the $\phi$-sensors, the $\phi$-sensor is divided in two parts.
The outer region starts at a radius of 17.25 mm and has approximately twice the
number of strips as the inner region. The strips in both regions make 
a $5^\circ$ stereo angle with respect to the radial to improve pattern 
recognition, and adjacent stations are placed with opposite angles with respect
to the radial. In order to fully cover the azimuthal angle with respect to the 
beam axis, the two detector halves overlap, as is shown in
Fig.~\ref{fig:velomodule}. 

For reason to minimize the amount of material between the particle vertices active detector
layers the detector is placed inside vacuum. To separate the primary beam vacuum
from the secondary vessel vacuum and shield the detector from RF pickup from the
beam, the sensors are separated from the beam vacuum by a thin aluminium foil.
Both the sensors and this commonly named RF-foil are contained inside a vacuum vessel.
During beam injection the two halves of the VELO are retracted 3 cm away from
the nominal beam position. The RF-foil is designed to minimize interactions.

\subsection{Magnet}

To provide a good momentum resolution, the LHCb experiment
utilizes a (dipole) magnet (see Fig.~\ref{fig:magnet}), which bends the tracks of charged particles.
The not superconducting magnet consists of two saddle-shaped coils.
These are placed mirror symmetrically, such that the gap left open by the magnet
is slightly larger than the LHCb acceptance, and the principal field component is 
vertical throughout the detector acceptance.

\begin{figure}[tb]
\begin{center}
\includegraphics[width=300px]{magnet}
\end{center}
\caption{\small The LHCb dipole magnet. The proton-proton interaction
region lies behind the magnet.}
\label{fig:magnet}
\end{figure}


The quantity important for momentum resolutions, and hence for the analysis
of $\chi_b \rightarrow \Upsilon \gamma$ channel, is the integrated magnetic field
the magnet delivers. For tracks passing through the entire tracking system this is
\cite{Alves:2008zz}:   

$$ \int Bdl = 4 Tm $$
making it possible to measure charged particles’ momenta up to 200 GeV within 0.5~\%
uncertainty.

\subsection{Inner tracker}

\begin{figure}[tb]
\begin{center}
\includegraphics[width=300px]{veloit}
\end{center}
\caption{\small Layout of the IT.}
\label{fig:veloit}
\end{figure}

To perform accurate momentum estimates, important for mass, angular and proper
time resolutions in the reconstruction of the 
$\chi_b \rightarrow \Upsilon \gamma$ channel, hit information downstream
of the magnet is required, which is provided by three tracking stations. Since 
the magnet bends particles in the horizontal direction perpendicular to the
beam pipe, the track density is largest in an elliptically shaped region around
the beam pipe. In order to have similar occupancies over the plane, a detector 
with finer detector granularity is required in this region. 
Therefore, the Inner Tracker (IT), 120 cm wide and 40 cm high, as shown in
Fig.~\ref{fig:veloit} , is located in the center of the three tracking stations.



Due to the high track density near the beam pipe, silicon strip detectors 
are used. The total active detector area covers 4.0 $m^2$ , consisting of 129024
readout strips of either 11 cm or 22 cm in length. To improve track
reconstruction, the four detector layers are arranged in an x-u-v-x geometry, 
in which the strips are vertical in the first and in the last layer, whereas the
other two (u, v) layers are rotated by stereo angles of $\pm5^\circ$ , providing the
sensitivity in the vertical direction.

The pitch of the single-sided $p^+$-on-n strips is 198 $\mum$. In order to have
similar performance in terms of signal-to-noise, the thickness of the sensors
is 320 $\mum$ for the single-sensor ladders below and above the beam pipe, and
410 $\mum$ for the double sensors at the sides of the beam pipe. The four layers
are housed in 4 boxes, which are placed such that they overlap. These overlaps
avoid gaps in the detector and, more importantly, make it possible to perform
alignment using reconstructed tracks.



\subsection{Outer tracker}

Similar to the IT, the Outer Tracker (OT) performs track measurements downstream of
the magnet, allowing to determine the momenta of charged particles. The OT covers the
outer region of the three tracking stations T1-T3.

Since the track density further away from the beam pipe is lower, straw tubes are used.
The total active area of one station is 5971$\times$4850 $mm^2$ , and the OT and the IT together
cover the full acceptance of the experiment. As is the case for the IT, these layers are also
arranged in an x-u-v-x geometry, as shown in Fig.~\ref{fig:veloot}.

\begin{figure}[tb]
\begin{center}
\includegraphics[width=300px]{veloot}
\end{center}
\caption{\small Layout of the OT.}
\label{fig:veloot}
\end{figure}

The OT is designed as an array of individual, gas-tight straw-tube modules.
Each module contains two layers of drift-tubes with an inner diameter of 4.9 mm. 
The front-end (FE) electronics measures the drift time of the ionization clusters
produced by charged particles traversing the straw tubes, digitizing it with respect to every
bunch crossing. Given the bunch crossing rate of 25 ns and the diameter of the tube, and in
order to guarantee a fast drift time (below 50 ns) and a sufficient drift-coordinate resolution
(200~$\mum$), a mixture of Argon (70\%) and CO2 (30\%) is used as counting gas.

\subsection{Tracker Turicensis}

To improve the momentum estimate of charged particles, track measurements are performed
before these enter the magnet. Therefore, the Tracker Turicensis (TT), a planar tracking
station, is located between the VELO and the LHCb dipole magnet. It is also used to
perform the track measurements of long lived neutral particles which decay after the VELO.
In addition, by using the weak magnetic field inside the tracker,
track information from the TT is used by the High Level Trigger to confirm candidates
between the VELO and the tracking stations.

In order to cover the full acceptance of the experiment, the TT is constructed 150 cm
wide and 130 cm high. It consists of four detector layers, with a total active
area of 8.4 $m^2$ , with 143360 readout channels, up to 38 cm in length. 
To improve track reconstruction, the four detector layers are arranged in two
pairs that are separated by approximately 27 cm along the LHCb beam axis.
And again, like the IT and the OT, the TT detection layers are in an x-u-v-x arrangement.

\begin{figure}[tb]
\begin{center}
\includegraphics[width=300px]{velott}
\end{center}
\caption{\small Layout of one of the stereo plane detector layers of the TT}
\label{fig:velott}
\end{figure}


The layout of one of the detector layers is illustrated in Fig.~\ref{fig:velott}.
Its basic building block is a half module that covers half the height of the
LHCb acceptance. It consists of a row of seven silicon sensors, named a ladder.
The silicon sensors for the TT are 500 $\mum$ thick, single sided $p^+$-on-n
sensors, as for the IT. They are 9.64~cm~$\times$~9.44~cm long and carry 512
readout strips with a strip pitch of 183 $\mum$.
    



\section{Particle identification}
\subsection{RICH system}
\subsection{Muon system}
\subsection{Calorimeter system}

\section{Trigger}
\subsection{L0 trigger}
\subsection{High level trigger}