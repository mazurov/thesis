\chapter{Introduction}
In 1977 a new family of heavy particles was discovered experimentally by the
Fermilab E288 experiment headed by Leon Lederman\cite{Herb:1977ek}. These
particles, named $\Upsilon$ mesons, consist of a pair of $b$- and
$\bar{b}$ quarks in a bound state. Their discovery was the first evidence of a type of
quarks, predicted in 1973 by Makoto Kobayashi and Toshihide Maskawa when they
tried to explain CP violation in the Standard Model~\cite{Kobayashi:1973fv}.
Later the evidence of b-quarks was confirmed by observation of B-mesons at
ARGUS\cite{Albrecht:1986nr} and CLEO\cite{Bebek:1987bp} experiments.
In the past, particles containing the b quark were largely studied at \lep, \tevatron and, 
more recently, in the B factories (\babar, \belle).
Nowadays, the LHCb experiment dominates this field by performing 
precise measurements of B hadrons spectroscopy, decays and CP violation. 

{\em{Bottomonium}} states are $b\bar{b}$ bound states. They are usually included in the
{\em{heavy quarkonia}} family together with {\em{charmonium}} $c\bar{c}$ bound
states, such as {\em{e.g.}} the $\jpsi$ particle. Being composed of a quark and 
an anti-quark, these states are mesons. The
study of quarkonia is very important because it provides a unique way to
understand and test the Quantum ChromoDynamics (QCD) theory. Meson properties, 
such as their masses, can be computed non-pertubatively using lenghty
Lattice QCD (LQCD) techniques, which solve the exact QCD equations by using 
numerical methods. However, the masses of quarkonia are large with respect to 
the typical hadronic energy scale. Therefore, the speed of heavy quarks 
inside their own quarkonia is non-relativistic and computations can be performed 
by using effective techniques, such as the Non-Relativistic QCD
(NRQCD, ~\cite{Dowdall:2011iy,Dowdall:2012ab}).
Besides assuming that heavy quarks in quarkonia are non-relativistic, 
these effective models suppose that they move in a static
potential\cite{Kulshreshtha:1984mw,Parmar:2010ii,shah}. In this sense, quarkonia in QCD 
are analogous to the hydrogen atom or the positronium in QED. 

The production mechanism of bottomonium states is not yet well understood.
Several models exist, such as the Colour Singlet Model (CSM, NLO CSM),
non-relativistic QCD expansion (NRQCD) with contributions from Colour Singlet and 
Colour Octet, and Colour Evaporation Model (CEM). None of these models succeeded in
explaining all experimental results on cross-section and polarization measurements.
More experimental inputs will be useful in resolving the theoretical models.

In this thesis a study of $\chi_b$ production is performed on data from proton-proton collisions, 
collected by the \lhcb experiment at the LHC at centre-of-mass energies 
\sqs=7 and 8\tev. The analyzed data set corresponds to a total integrated luminosity
of 3\invfb. The $\chi_b$ mesons were reconstructed by using 
$\chi_b(nP)\rightarrow\Upsilon(mS)\gamma$ decays. Single differential
production cross sections, relative to the production cross-sections of
$\Upsilon$ mesons, are measured as function of $\Upsilon$ transverse momentum.
A measurement of the $\chi_b(3P)$ mass, which was recently observed by the 
ATLAS~\cite{Aad:2011ih}, D0~\cite{Abazov:2012gh} and
LHCb~\cite{LHCb-CONF-2012-020} collaborations, is also performed.

The analysis performed in this thesis extends the previous \lhcb
study\cite{LHCb-PAPER-2012-015}, which reported a measurement of the $\chi_b$
production cross-section, and subsequent decay into $\Y1S \gamma$, relative
to the \Y1S production. That measurement was performed on 32\invpb data set,
collected at a centre-of-mass energies \sqs=7\tev in 2010,  in a region defined by
transverse momentum $6  < p_{T}^{\Y1S} < 15 \gevc$ and rapidity  $2.0 <
y^{\Y1S} < 4.5$. 

This analysis improves significantly the statistical precision of the previous work and 
adds more decays and transverse momentum regions. The
\lhcb detector design allows to perform measurements in $\Upsilon$ rapidity and 
transverse momentum regions, which are complementary to the ones exploited 
by other LHC experiments. 

Muon triggering and offline muon identification are fundamental for this analysis, and 
for the physics program of LHCb, with the software High Level Trigger (HLT) being 
a crucial player. In this thesis the author presents the tool, which helps to
analyze and improve the performance of HLT software. A few spots poorly performing 
in the HLT code were identified and fixed by using this tool.

The performance tool was reported at the 19th International
Conference on Computing in High Energy and Nuclear Physics (New-York, 2012) and
the corresponding proceeding was published\cite{aprofiler}. The results on
\chib production were regularly presented at the LHCb bottomonium working group,
an internal document was prepared by the author, is currently under review and will form 
the basis of a future LHCb publication.

This thesis is organized as follows. 
Chapter 2 briefly reviews the quarkonium production mechanisms. Chapter 3
outlines the LHCb experiment design. In Chapter 4 the software performance tool 
is presented, and Chapter 5 shows the $\chi_b$ analysis procedure and
results of cross-section and \chibThreeP mass measurements. The thesis results are 
summarized in the Conclusion. 
