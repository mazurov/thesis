In 1977 a new family of heavy particles was discovered experimentally by the
Fermilab E288 experiment headed by Leon Ledereman\cite{Herb:1977ek}. These
particles, named as $\Upsilon$ mesons, are the pair of $b$- and
$\bar{b}$-quarks and the discovery was the first evidence of this type of
quarks, predicted in 1973 by Makoto Kobayashi and Toshihide Maskawa when they
tried to explain CP violation in the Standard Model~\cite{Kobayashi:1973fv}.
Later the evidence of b-quarks was confirmed by observation of B-mesons at
ARGUS\cite{Albrecht:1986nr} and CLEO\cite{Bebek:1987bp} experiments.

The Standard Model (SM) is the theory which describes the electromagnetic,
weak and strong interactions between elementary particles. The model describes
a wide variety of subatomic phenomena involving known elementary particles and
has been confirmed with high precision measurements. This theory is based on
the quark model started in the 1960s\cite{GellMann:1964nj,Zweig:1981pd}, few
years before the experimental confirmation of the existence of quarks. The
theory says that quarks are elementary particles which interact due to the
strong interaction. Because of a phenomenon called color confinement, they are
never been directly observed,  but can be found within composite particles
called hadron. There are two types of hadrons named as baryons (qqq), made of
three quarks or anti- quarks, and mesons ($q\bar{q}$), made of one quark and
anti-quark. The quark model is composed of 6 different types of quarks known as
flavors: up (u), down (d), strange (s), charm (c), bottom or beauty (b), top or
truth (t). This thesis is focused especially on the b quark. In the past the b
quark was largely studied at \lep then in B factories (\babar, \belle).
Nowadays, this field is covered by the LHCb experiment designed to perform
precise measurements on B-meson decays.

A bottomonium state is a $b\bar{b}$ bound state. This state belongs to the
quarkonia family together with the charmonium $c\bar{c}$ bound
state such as $\jpsi$ particle for example. These states are mesons, or more
precisely B-mesons, because they are composed of a quark and anti-quark. The
study of quarkonia is very important because it provides a unique way to
understand and test the Quantum ChromoDynamics (QCD) theory. Properties of
mesons, like their masses, can be computed non-pertubatively using lenghty
Lattice QCD (LQCD) techniques. But, due to their huge masses, the speed of c
and b quarks is non-relativistic in their own quarkonia, then the computation
can be approximated. This effective technique, called Non-Relativistic QCD
(NRQCD), provides a good non-perturbative test for LQCD and can predict
bottomonium masses~\cite{Dowdall:2011iy,Dowdall:2012ab}.

There is exists the technique based on the fact that c and b quarks are non-
relativistic and supposes that they move in a static
potential\cite{Kulshreshtha:1984mw,Parmar:2010ii,shah}. This potential have two
parts, the first is Coulombic which corresponds to a one-gluon exchange between
the quark and its anti-quark, the second represents the color confinement part
of the potential. To make an analogy, in this model, quarkonia is like an
hydrogen atom but for strong interaction because only the shape of the
potential changes.

There is a wide variety of bottomonium states, each differing from other by quantum
numbers: the principal quantum number ($n$), the relative angular momentum
between the quarks ($L$), the spin combination of the two quarks ($S$) and the total
angular momentum ($J$) with $J = L + S$. In particle physics, the notation $J^{PC}$ is
often used, where $P$ is the parity number and $C$ is the charge conjugation number.
For the bottomonium states, they are defined as $P=(-)^{L+1}$ and $C=(-)^{L+S}$.

% All the known bottomonium states and their quantum numbers are represented on
% FIG. 1. The L = 0 and L = 1 states are respectively called S-wave and P-wave,
% for example Υ(1S) is a S-wave and hb(2P) is a P-wave. The number n orders
% states with same other quantum numbers from lowest to highest masses such as
% for Υ(1S), Υ(2S), Υ(3S) and Υ(4S) with n respectively equal to 1, 2, 3 and 4.
% When the conditions L = 1 and S = 1 are satisfied, J takes the value 0, 1 or 2
% causing mass level splitting, this is called the spin-orbit coupling. Thus,
% each χb states has three sub-states indexed by the value of the quantum number
% J. As an example, in the case of χb(1P), we have χb0(1P), χb1(1P) and χb2(1P)
% corresponding respectively to J = 0, 1 and 2. It is also possible to observe
% radiative transitions, i.e. transitions involving a photon in the final state,
% from one bottomonium state to another. The selection rules are the same as for
% the hydrogen atom energy states. The electric dipole transition (E1) is the
% leading order transition, it changes the relative angular momentum ∆L = ±1 but
% not the spin ∆S = 0. The magnetic dipole transition (M1) is a next to leading
% order transition and therefore it’s very rare, it changes the spin ∆S = ±1 but
% not the relative angular momentum ∆L = 0. The last bottomonium state discovered
% was the χb(3P) by the ATLAS experiment [6] followed by D0 [7] and LHCb
% experiment [8] and it was also the first particle discovered at the LHC.
% However, because the resolution on the invariant mass of the χb(3P) was large,
% it was not possible to see its sub-states. The aim of this study is to
% elaborate a strategy in order to provide a first measurement of the three χb(3P)
% sub-states masses. For this, I will use the radiative decays (E1 transitions)
% between one of the three χb states to the Υ(1S) (see FIG. 1) with a converted
% photon because it provides a better resolution on the χb(3P) invariant mass.

