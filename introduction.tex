\chapter{Introduction}
In 1977 a new family of heavy particles was discovered experimentally by the
Fermilab E288 experiment headed by Leon Ledereman\cite{Herb:1977ek}. These
particles, named as $\Upsilon$ mesons, are the pair of $b$- and
$\bar{b}$-quarks and the discovery was the first evidence of this type of
quarks, predicted in 1973 by Makoto Kobayashi and Toshihide Maskawa when they
tried to explain CP violation in the Standard Model~\cite{Kobayashi:1973fv}.
Later the evidence of b-quarks was confirmed by observation of B-mesons at
ARGUS\cite{Albrecht:1986nr} and CLEO\cite{Bebek:1987bp} experiments.

This thesis is focused especially on the b quark. In the past the b
quark was largely studied at \lep, \tevatron then in B factories (\babar, \belle).
Nowadays, this field is covered by the LHCb experiment designed to perform
precise measurements on B-meson decays.

A bottomonium state is a $b\bar{b}$ bound state. This state belongs to the
quarkonia family together with the charmonium $c\bar{c}$ bound
state, such as $\jpsi$ particle for example. These states are mesons, or more
precisely B-mesons, because they are composed of a quark and anti-quark. The
study of quarkonia is very important because it provides a unique way to
understand and test the Quantum ChromoDynamics (QCD) theory. Properties of
mesons, like their masses, can be computed non-pertubatively using lenghty
Lattice QCD (LQCD) techniques. But, due to their huge masses, the speed of c
and b quarks is non-relativistic in their own quarkonia, then the computation
can be approximated. This effective technique, called Non-Relativistic QCD
(NRQCD), provides a good non-perturbative test for LQCD and can predict
bottomonium masses~\cite{Dowdall:2011iy,Dowdall:2012ab}.

There is exists the model based on the fact that c and b quarks are
non-relativistic and supposes that they move in a static
potential\cite{Kulshreshtha:1984mw,Parmar:2010ii,shah}. To make an analogy,
in this model, quarkonia is like an hydrogen atom but for strong interaction.

The production mechanism of bottomonia ̄ states is not yet well understood.
There are several models exist, such as Colour Singlet Model (CSM, NLO CSM),
non-relativistic QCD expansion (NRQCD) with contributions from Colour Octet
Mechanism and Colour Evaporation Model (CEM). None of these models succeeded in
explaining all experimental results on cross-section and polarization measurements.
More experimental inputs will be useful in resolving the theoretical models.

In this thesis a study of $\chi_b$ production is performed on 3\invfb data set,
collected at a centre-of-mass energies \sqs=7 and 8\tev at \lhcb experiment,
using $\chi_b(nP)\rightarrow\Upsilon(mS)\gamma$ decays. A single differential
production cross sections, relative to the production cross-sections of
$\Upsilon$ mesons are measured as functions of $\Upsilon$ transverse momentum.
A measurement of the $\chi_b(3P)$ mass, which was recently observed at
ATLAS~\cite{Aad:2011ih}, D0~\cite{Abazov:2012gh} and
LHCb~\cite{LHCb-CONF-2012-020} collaborations, is also performed.

The author presents an improvement of the previous \lhcb
study\cite{LHCb-PAPER-2012-015} which reported a measurement of the $\chi_b$
production cross-section, and subsequent decay into $\Y1S \gamma$, relative
to the \Y1S production. That measurement was performed on 32\invpb data set,
collected at a centre-of-mass energies \sqs=7 in 2010,  in a region defined by
transverse momentum $6  < p_{T}^{\Y1S} < 15 \gevc$ and rapidity  $2.0 <
y^{\Y1S} < 4.5$. 

The results have a higher statistical precision than in previous work. The
author analyzes considerably more decays and transverse momentum regions. The
uniqueness of results can be explained by the \lhcb detector design, which
allows to perform measurements in much higher $\Upsilon$ rapidity and
transverse momentum ranges than at other experiments.

Muon triggering and offline muon identification are fundamental requirements of
the LHCb experiment. The software High Level Trigger (HLT) is the main part of
this system. In this thesis the author presents the tool, which helps to
analyze and improve a software performance. A few performance critical places
in the HLT code were identified and fixed by using this tool.

The thesis was done within the \lhcb physical program. The 
performance tool was reported at the 19th International
Conference on Computing in High Energy and Nuclear Physics (New-York, 2012) and
the corresponding proceeding was published\cite{aprofiler}. The results on
\chib production were regularly presented at the LHCb bottomonium working group,
the internal document was prepared by author and is a subject for future
publication.

Chapter 2 reviews the production mechanisms and experimental status. Chapter 3
outlines the LHCb experiment design. In Chapter 4 the tool that analyze a
software performance is presented, and Chapter 5 shows the procedure and
results of cross-section and \chibThreeP mass measurements. Conclusion
summarizes the thesis results.
