In 1977 a new family of heavy particles was discovered experimentally by the
Fermilab E288 experiment headed by Leon Ledereman\cite{Herb:1977ek}. These
particles, named as $\Upsilon$ mesons, are the pair of $b$- and
$\bar{b}$-quarks and the discovery was the first evidence of this type of
quarks, predicted in 1973 by Makoto Kobayashi and Toshihide Maskawa when they
tried to explain CP violation in the Standard Model~\cite{Kobayashi:1973fv}.
Later the evidence of b-quarks was confirmed by observation of B-mesons at
ARGUS\cite{Albrecht:1986nr} and CLEO\cite{Bebek:1987bp} experiments.

The Standard Model (SM) is the theory which describes the electromagnetic,
weak and strong interactions between elementary particles. The model describes
a wide variety of subatomic phenomena involving known elementary particles and
has been confirmed with high precision measurements. This theory is based on
the quark model started in the 1960s\cite{GellMann:1964nj,Zweig:1981pd}, few
years before the experimental confirmation of the existence of quarks. The
theory says that quarks are elementary particles which interact due to the
strong interaction. Because of a phenomenon called color confinement, they are
never been directly observed,  but can be found within composite particles
called hadron. There are two types of hadrons named as baryons (qqq), made of
three quarks or anti- quarks, and mesons ($q\bar{q}$), made of one quark and
anti-quark. The quark model is composed of 6 different types of quarks known as
flavors: up (u), down (d), strange (s), charm (c), bottom or beauty (b), top or
truth (t). This thesis is focused especially on the b quark. In the past the b
quark was largely studied at \lep then in B factories (\babar, \belle).
Nowadays, this field is covered by the LHCb experiment designed to perform
precise measurements on B-meson decays.

A bottomonium state is a $b\bar{b}$ bound state. This state belongs to the
quarkonia family together with the charmonium $c\bar{c}$ bound
state, such as $\jpsi$ particle for example. These states are mesons, or more
precisely B-mesons, because they are composed of a quark and anti-quark. The
study of quarkonia is very important because it provides a unique way to
understand and test the Quantum ChromoDynamics (QCD) theory. Properties of
mesons, like their masses, can be computed non-pertubatively using lenghty
Lattice QCD (LQCD) techniques. But, due to their huge masses, the speed of c
and b quarks is non-relativistic in their own quarkonia, then the computation
can be approximated. This effective technique, called Non-Relativistic QCD
(NRQCD), provides a good non-perturbative test for LQCD and can predict
bottomonium masses~\cite{Dowdall:2011iy,Dowdall:2012ab}.

There is exists the model based on the fact that c and b quarks are
non-relativistic and supposes that they move in a static
potential\cite{Kulshreshtha:1984mw,Parmar:2010ii,shah}. This potential have two
parts, the first is Coulombic which corresponds to a one-gluon exchange between
the quark and its anti-quark and proportional to $1/r$, where r is the
effective radius of the quarkonium state, the second represents the color
confinement part of the potential and proportional to $r$. To make an analogy,
in this model, quarkonia is like an hydrogen atom but for strong interaction
because only the shape of the potential changes.

The production mechanism of bottomonia ̄ states is not yet well understood.
There are several models exist, such as Colour Singlet Model (CSM, NLO CSM),
non-relativistic QCD expansion (NRQCD) with contributions from Colour Octet
Mechanism and Colour Evaporation Model (CEM). None of these models succeeded in
explaining all experimental results on cross-section and polarization measurements.
More experimental inputs will be useful in resolving the theoretical models.

There is a wide variety of bottomonium states, each differing from other by quantum
numbers: the principal quantum number ($n$), the relative angular momentum
between the quarks ($L$), the spin combination of the two quarks ($S$) and the total
angular momentum ($J$) with $J = L + S$. In particle physics, the notation $J^{PC}$ is
often used, where $P$ is the parity number and $C$ is the charge conjugation number.
For the bottomonium states, they are defined as $P=(-)^{L+1}$ and $C=(-)^{L+S}$.

\begin{figure}[H]
  \setlength{\unitlength}{1mm}
  \centering
  \resizebox{\textwidth}{!}{
  \begin{picture}(150,80)
  \put(0,0){\includegraphics*[width=150mm, height=80mm]{bottomonium}}
  \put(3,25){\Large \begin{sideways}$m(b\bar{b}) \left[\gevcc\right]$\end{sideways}}
  \end{picture}
  }
\caption{Observed (blue) and theoretically predicted (red) bottomonium states}
\label{fig:bottomonium}
\end{figure} 



All bottomonium states and their quantum numbers are shown
in~\Cref{fig:bottomonium}. The $L = 0$ and $L = 1$ states are respectively
called S-wave and P-wave. For example, \Y1S is a S-wave and \chiboneOneP is a
P-wave. The principal quantum number $n$ orders states from lowest to highest
masses, such as for $\Upsilon(nS)$, where $n$ equals to 1, 2, 3 and 4. When the
conditions $L = 1$ and $S = 1$ are satisfied, J takes the value 0, 1 or 2
causing mass level splitting, which is called the spin-orbit coupling. Thus,
each \chib states has three sub-states indexed by the value of the quantum
number J. For example, $\chi_b(1P)$ state has three sub-states $\chi_{bJ}(1P)$,
where n equals to 1, 2 and 3.

Radiative transition from one bottomonium state to another is possible to
observe. This transition involve a photon in the final state. The selection
rules are the same as for the hydrogen atom energy states. The electric dipole
transition is the leading order transition, which changes the relative angular
momentum $\Delta L  = \pm 1$ but not the spin $\Delta S = 0$. The magnetic
dipole transition is a next to leading order transition and therefore is very
rare. This transition changes the spin $\Delta S = \pm 1$ but not the relative
angular momentum $\Delta L = 0$.

In this thesis a study of $\chi_b$ production is performed on 3\invfb data set,
collected at a centre-of-mass energies \sqs=7 and 8\tev at \lhcb experiment,
using $\chi_b(nP)\rightarrow\Upsilon(mS)\gamma$ decays. A single differential
production cross sections, relative to the production cross-sections of
$\Upsilon$ mesons are measured as functions of $\Upsilon$ transverse momentum.
A measurement of the $\chi_b(3P)$ mass, which was recently observed at
ATLAS~\cite{Aad:2011ih}, D0~\cite{Abazov:2012gh} and
LHCb~\cite{LHCb-CONF-2012-020} collaborations, is also performed.

The author presents an improvement of the previous \lhcb
study\cite{LHCb-PAPER-2012-015} which reported a measurement of the $\chi_b$
production cross-section, and subsequent decay into $\Y1S \gamma$, relative
to the \Y1S production. That measurement was performed on 32\invpb data set,
collected at a centre-of-mass energies \sqs=7 in 2010,  in a region defined by
transverse momentum $6  < p_{T}^{\Y1S} < 15 \gevc$ and rapidity  $2.0 <
y^{\Y1S} < 4.5$. 

The results have a higher statistical precision than in previous work. The
author analyzes considerably more decays and transverse momentum regions. The
uniqueness of results can be explained by the \lhcb detector design, which
allows to perform measurements in much higher $\Upsilon$ rapidity and
transverse momentum ranges than at other experiments.

Muon triggering and offline muon identification are fundamental requirements of
the LHCb experiment. The software High Level Trigger (HLT) is the main part of
this system. In this thesis the author presents the tool, which helps to
analyze and improve a software performance. A few performance critical places
in the HLT code were identified and fixed by using this tool.

The thesis was done within the \lhcb physical program. The 
performance tool was reported at the 19th International
Conference on Computing in High Energy and Nuclear Physics (New-York, 2012) and
the corresponding proceeding was published\cite{aprofiler}. The results on
\chib production were regularly presented at the LHCb bottomonium working group,
the internal document was prepared by author and is a subject for future
publication.

Chapter 2 reviews the production mechanisms and experimental status. Chapter 3
outlines the LHCb experiment design. In Chapter 4 the tool that analyze a
software performance is presented, and Chapter 5 shows the procedure and
results of cross-section and \chibThreeP mass measurements. Conclusion
summarizes the thesis results.
