\chapter{Review of production mechanism and experimental status}

The Standard Model (SM) is the theory which describes the electromagnetic,
weak and strong interactions between elementary particles. The model describes
a wide variety of subatomic phenomena involving known elementary particles and
has been confirmed with high precision measurements. This theory is based on
the quark model started in the 1960s\cite{GellMann:1964nj,Zweig:1981pd}, few
years before the experimental confirmation of the existence of quarks. The
theory says that quarks are elementary particles which interact due to the
strong interaction. Because of a phenomenon called color confinement, they are
never been directly observed,  but can be found within composite particles
called hadron. There are two types of hadrons named as baryons (qqq), made of
three quarks or anti- quarks, and mesons ($q\bar{q}$), made of one quark and
anti-quark. The quark model is composed of 6 different types of quarks known as
flavors: up (u), down (d), strange (s), charm (c), bottom or beauty (b), top or
truth (t). 


There is a wide variety of bottomonium states, each differing from other by quantum
numbers: the principal quantum number ($n$), the relative angular momentum
between the quarks ($L$), the spin combination of the two quarks ($S$) and the total
angular momentum ($J$) with $J = L + S$. In particle physics, the notation $J^{PC}$ is
often used, where $P$ is the parity number and $C$ is the charge conjugation number.
For the bottomonium states, they are defined as $P=(-)^{L+1}$ and $C=(-)^{L+S}$.

\begin{figure}[H]
  \setlength{\unitlength}{1mm}
  \centering
  \resizebox{\textwidth}{!}{
  \begin{picture}(150,80)
  \put(0,0){\includegraphics*[width=150mm, height=80mm]{bottomonium}}
  \put(3,25){\Large \begin{sideways}$m(b\bar{b}) \left[\gevcc\right]$\end{sideways}}
  \end{picture}
  }
\caption{Observed (blue) and theoretically predicted (red) bottomonium states}
\label{fig:bottomonium}
\end{figure} 



All bottomonium states and their quantum numbers are shown
in~\Cref{fig:bottomonium}. The $L = 0$ and $L = 1$ states are respectively
called S-wave and P-wave. For example, \Y1S is a S-wave and \chiboneOneP is a
P-wave. The principal quantum number $n$ orders states from lowest to highest
masses, such as for $\Upsilon(nS)$, where $n$ equals to 1, 2, 3 and 4. When the
conditions $L = 1$ and $S = 1$ are satisfied, J takes the value 0, 1 or 2
causing mass level splitting, which is called the spin-orbit coupling. Thus,
each \chib states has three sub-states indexed by the value of the quantum
number J. For example, $\chi_b(1P)$ state has three sub-states $\chi_{bJ}(1P)$,
where n equals to 1, 2 and 3.

Radiative transition from one bottomonium state to another is possible to
observe. This transition involve a photon in the final state. The selection
rules are the same as for the hydrogen atom energy states. The electric dipole
transition is the leading order transition, which changes the relative angular
momentum $\Delta L  = \pm 1$ but not the spin $\Delta S = 0$. The magnetic
dipole transition is a next to leading order transition and therefore is very
rare. This transition changes the spin $\Delta S = \pm 1$ but not the relative
angular momentum $\Delta L = 0$.