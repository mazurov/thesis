\section[Systematic]{Systematic Uncertainties}
\label{sec:syst}

Since this analysis measures the fraction of $\Upsilon(nS)$ particles
originating from $\chi_b$ decays, most systematic uncertainties cancel in the
ratio and only residual effects need to be taken into account. Systematic
uncertainties can be grouped according to their contribution to the terms
of~\Cref{eqn:master}. Systematic uncertainties on the event yields are mostly
due to the fit model of $\Upsilon$ and $\chi_b$ invariant masses, while the
ones on the efficiency are due to the photon reconstruction and the unknown
initial polarization of $\chi_b$ and $\Upsilon$ particles. The systematic uncertainties due to 
polarization of inclusive $\Upsilon$ mesons are considered to be
small~\cite{Aaij:2013yaa,Aaij:2014nwa}.
 

\subsection{Uncertainties related to the fit model}

The uncertainty related to the modeling of the $\Upsilon$ invariant mass
distribution has been estimated by following previous
studies~\cite{Aaij:2013yaa}. An uncertainty of 0.7\% has been assigned to the
yields of $\Upsilon(nS)$ mesons.

In the fit model of the $\chi_b$ invariant masses, several sources need to be
taken into account. Firstly, the relative proportion of spin-1 and spin-2
states, which is kept fixed in the fit to values close to 0.5, predicted by
theory, is varied from 0.3 to 0.7 as it was discussed in~\Cref{sec:ratio}.


Tables~\ref{tab:syst:lambda_ups1s},\ref{tab:syst:lambda_ups2s} and
\ref{tab:syst:lambda_ups3s} report the relative variation in percent of the
$\chi_b$ yields as function of $\lambda$, the relative proportion of the two
$\chi_b$ states, for all examined decays, in each bin of transverse momentum
for $\chi_b$ decays into $\Upsilon(1S)$, $\Upsilon(2S)$ and $\Upsilon(3S)$,
respectively. We take as systematic error in each $p_T$ bin the maximum
variation of the $\chi_b$ yields with respect to the nominal fit.

Another source of systematic is due to the variation of the $\chi_b$ masses as
function of $p_T(\Upsilon)$, observed in~\Cref{sec:chib}. We repeat the fits by
taking the minimum and maximum values of the $\chi_b$ masses and take the
maximum difference in the yields as systematic uncertainty. The resulting
uncertainties are reported
in~\Cref{tab:syst:m_ups1s,tab:syst:m_ups2s,tab:syst:m3p_ups3s} for $\chi_b$
decays into $\Upsilon(1S)$, $\Upsilon(2S)$ and $\Upsilon(3S)$, respectively.
 
Other uncertainties due to parameters taken from PDG (e.g. mass differences)
are assumed to be negligible.
 
\subsubsection{Uncertainties related to Data-MonteCarlo differences in resolution}
 
In the $\chi_b$ fits, the resolution of the CrystalBall functions, determined
from simulation, has been scaled by a factor 1.17 in order to account for data
--- MonteCarlo differences. This factor was obtained by fitting a histogram,
that stores the ratio between data and MonteCarlo resolution, by a constant
function. The result of fit is shown in~\Cref{fig:syst:ratio_data_mc_sigma}.
 
% \begin{figure}[H]
  \setlength{\unitlength}{1mm}
  \centering
  \begin{picture}(75, 60)
  \put(0,0){
    \includegraphics*[width=75mm, height=60mm]{syst/sigma_b1_1p}
  }
  \put(50,0){$p_T^{\Y1S} \left[\gevc\right]$}
  \put(0,20){\begin{sideways} $\sigma_{\chiboneOneP} \left[\gevcc\right]$ \end{sideways}}
  
  \put(45,50){\includegraphics*[width=4mm, height=2mm]{blue}}
  \put(45,45){\includegraphics*[width=4mm, height=2mm]{red}}
  \put(49,50){\sqs=7\tev}
  \put(49,45){\sqs=8\tev}



  \end{picture}
  \label{fig:syst:data_sigma}
  \caption{\small \chiboneOneP yield resolution in $\chi_b \to \Y1S \gamma$ decay}
\end{figure}
\begin{figure}[H]
  \setlength{\unitlength}{1mm}
  \centering
  \begin{picture}(150, 60)
  \put(0,0){
    \includegraphics*[width=75mm, height=60mm]{syst/sigma_data_mc_2011}
  }
  \put(75,0){
    \includegraphics*[width=75mm, height=60mm]{syst/sigma_data_mc_2012}
  }

  \put(50,0){$p_T^{\Y1S} \left[\gevc\right]$}
  \put(125,0){$p_T^{\Y1S} \left[\gevc\right]$}
  
  \put(0,10){\begin{sideways} $\sigma_{\chiboneOneP}^{data}/\sigma_{\chiboneOneP}^{MC} \left[\gevcc\right]$ \end{sideways}}
  \put(75,10){\begin{sideways} $\sigma_{\chiboneOneP}^{data}/\sigma_{\chiboneOneP}^{MC} \left[\gevcc\right]$ \end{sideways}}

  \put(45,50){\sqs=7\tev}
  \put(120,50){\sqs=8\tev}




  \end{picture}
  \caption{\small Ratio between  \chiboneOneP yield resolution in data and \chiboneOneP
  yield resolution in MonteCarlo in $\chiboneOneP \to \Y1S \gamma$ decay. The black
  line on the plot shows the result of histogram fit by constant function.
  }
  \label{fig:syst:ratio_data_mc_sigma}
\end{figure}

We estimate the systematic uncertainty due to this assumption by repeating the
fits with scaling the $\sigma$ parameter with maximum and minimum values of
the scaling factor obtained from the fit in Figure~\ref{fig:syst:ratio_data_mc_sigma}.
Results are shown in Table~\ref{tab:syst:sigma_ups1s}.



\subsection{Summary of systematic uncertainties on the \chib yields fractions in
$\chib \to \Upsilon \gamma$ decay related to \chib fit model}
\begin{table}[H]
\centering
\caption{\small $\chi_b$ yields systematic uncertainties (\%) related to the Data-MonteCarlo resolution difference in the fit model for $\chi_b(1,2,3P) \to \OneS \gamma$ decays}
\subtable[$6 < p_T^{\Y1S} < 10 \gevc$] {
\scalebox{0.65}{
\begin{tabular}{lrrrrrrrrrrrr}\toprule
 & \multicolumn{12}{c}{$\Upsilon(1S)$ transverse momentum intervals, \gevc}\\
 & \multicolumn{6}{c}{6 -- 8} & \multicolumn{6}{c}{8 -- 10}\\
\cmidrule(r){2-7}\cmidrule(r){8-13}
 & \multicolumn{3}{c}{\sqs = 7\tev} & \multicolumn{3}{c}{\sqs = 8\tev} & \multicolumn{3}{c}{\sqs = 7\tev} & \multicolumn{3}{c}{\sqs = 8\tev}\\
\cmidrule(r){2-4}\cmidrule(r){5-7}\cmidrule(r){8-10}\cmidrule(r){11-13}
 & \multicolumn{1}{c}{$N_{\chi_{b}(1P)}$} & \multicolumn{1}{c}{$N_{\chi_{b}(2P)}$} & \multicolumn{1}{c}{$N_{\chi_{b}(3P)}$} & \multicolumn{1}{c}{$N_{\chi_{b}(1P)}$} & \multicolumn{1}{c}{$N_{\chi_{b}(2P)}$} & \multicolumn{1}{c}{$N_{\chi_{b}(3P)}$} & \multicolumn{1}{c}{$N_{\chi_{b}(1P)}$} & \multicolumn{1}{c}{$N_{\chi_{b}(2P)}$} & \multicolumn{1}{c}{$N_{\chi_{b}(3P)}$} & \multicolumn{1}{c}{$N_{\chi_{b}(1P)}$} & \multicolumn{1}{c}{$N_{\chi_{b}(2P)}$} & \multicolumn{1}{c}{$N_{\chi_{b}(3P)}$}\\
\midrule
$\frac{\sigma_{\chiboneOneP}^{data}}{\sigma_{\chiboneOneP}^{MC}} = 1.13$ & 2.7 & 1.2 & --- & 2.6 & 0.5 & --- & 2.4 & 1.3 & --- & 2.5 & 1.5 & ---\\
$\frac{\sigma_{\chiboneOneP}^{data}}{\sigma_{\chiboneOneP}^{MC}} = 1.20$ & -3.8 & -0.6 & --- & -3.7 & 1.3 & --- & -3.3 & -1.5 & --- & -3.3 & -1.8 & ---\\
\bottomrule
\end{tabular}
} % scalebox

} % subtable
\subtable[$10 < p_T^{\Y1S} < 18 \gevc$] {
\scalebox{0.65}{
\begin{tabular}{lrrrrrrrrrrrr}\toprule
 & \multicolumn{12}{c}{$\Upsilon(1S)$ transverse momentum intervals, \gevc}\\
 & \multicolumn{6}{c}{10 -- 14} & \multicolumn{6}{c}{14 -- 18}\\
\cmidrule(r){2-7}\cmidrule(r){8-13}
 & \multicolumn{3}{c}{\sqs = 7\tev} & \multicolumn{3}{c}{\sqs = 8\tev} & \multicolumn{3}{c}{\sqs = 7\tev} & \multicolumn{3}{c}{\sqs = 8\tev}\\
\cmidrule(r){2-4}\cmidrule(r){5-7}\cmidrule(r){8-10}\cmidrule(r){11-13}
 & \multicolumn{1}{c}{$N_{\chi_{b}(1P)}$} & \multicolumn{1}{c}{$N_{\chi_{b}(2P)}$} & \multicolumn{1}{c}{$N_{\chi_{b}(3P)}$} & \multicolumn{1}{c}{$N_{\chi_{b}(1P)}$} & \multicolumn{1}{c}{$N_{\chi_{b}(2P)}$} & \multicolumn{1}{c}{$N_{\chi_{b}(3P)}$} & \multicolumn{1}{c}{$N_{\chi_{b}(1P)}$} & \multicolumn{1}{c}{$N_{\chi_{b}(2P)}$} & \multicolumn{1}{c}{$N_{\chi_{b}(3P)}$} & \multicolumn{1}{c}{$N_{\chi_{b}(1P)}$} & \multicolumn{1}{c}{$N_{\chi_{b}(2P)}$} & \multicolumn{1}{c}{$N_{\chi_{b}(3P)}$}\\
\midrule
$\frac{\sigma_{\chiboneOneP}^{data}}{\sigma_{\chiboneOneP}^{MC}} = 1.13$ & 2.2 & 1.8 & 4.3 & 2.2 & 1.5 & 6.0 & 1.4 & 2.6 & 1.8 & 1.5 & 2.0 & 2.9\\
$\frac{\sigma_{\chiboneOneP}^{data}}{\sigma_{\chiboneOneP}^{MC}} = 1.20$ & -2.9 & -2.4 & -5.7 & -3.1 & -1.2 & -11.1 & -1.8 & -3.3 & -2.3 & -1.9 & -2.7 & -3.7\\
\bottomrule
\end{tabular}
} % scalebox

} % subtable
\subtable[$18 < p_T^{\Y1S} < 40 \gevc$] {
\scalebox{0.65}{
\begin{tabular}{lrrrrrrrrrrrr}\toprule
 & \multicolumn{12}{c}{$\Upsilon(1S)$ transverse momentum intervals, \gevc}\\
 & \multicolumn{6}{c}{18 -- 22} & \multicolumn{6}{c}{22 -- 40}\\
\cmidrule(r){2-7}\cmidrule(r){8-13}
 & \multicolumn{3}{c}{\sqs = 7\tev} & \multicolumn{3}{c}{\sqs = 8\tev} & \multicolumn{3}{c}{\sqs = 7\tev} & \multicolumn{3}{c}{\sqs = 8\tev}\\
\cmidrule(r){2-4}\cmidrule(r){5-7}\cmidrule(r){8-10}\cmidrule(r){11-13}
 & \multicolumn{1}{c}{$N_{\chi_{b}(1P)}$} & \multicolumn{1}{c}{$N_{\chi_{b}(2P)}$} & \multicolumn{1}{c}{$N_{\chi_{b}(3P)}$} & \multicolumn{1}{c}{$N_{\chi_{b}(1P)}$} & \multicolumn{1}{c}{$N_{\chi_{b}(2P)}$} & \multicolumn{1}{c}{$N_{\chi_{b}(3P)}$} & \multicolumn{1}{c}{$N_{\chi_{b}(1P)}$} & \multicolumn{1}{c}{$N_{\chi_{b}(2P)}$} & \multicolumn{1}{c}{$N_{\chi_{b}(3P)}$} & \multicolumn{1}{c}{$N_{\chi_{b}(1P)}$} & \multicolumn{1}{c}{$N_{\chi_{b}(2P)}$} & \multicolumn{1}{c}{$N_{\chi_{b}(3P)}$}\\
\midrule
$\frac{\sigma_{\chiboneOneP}^{data}}{\sigma_{\chiboneOneP}^{MC}} = 1.13$ & 1.2 & 1.6 & 1.7 & 1.2 & 1.7 & 3.0 & 0.9 & 3.7 & 5.6 & 1.3 & 1.1 & 2.3\\
$\frac{\sigma_{\chiboneOneP}^{data}}{\sigma_{\chiboneOneP}^{MC}} = 1.20$ & -1.6 & -2.1 & -2.3 & -1.6 & -2.3 & -3.4 & -1.7 & -1.4 & -0.7 & -1.8 & -1.5 & -3.2\\
\bottomrule
\end{tabular}
} % scalebox

} % subtable
\label{tab:syst:sigma_ups1s}
\end{table}

\begin{table}[H]
\centering
\caption{\small $\chi_b$ yields systematic uncertainties (\%)  related to $\lambda$ values in the fit model for  $\chi_b(1,2,3P) \to \OneS \gamma$ decays}
\subtable[$6 < p_T^{\Y1S} < 10 \gevc$] {
\scalebox{0.65}{
\begin{tabular}{lrrrrrrrrrrrr}\toprule
 & \multicolumn{12}{c}{$\Upsilon(1S)$ transverse momentum intervals, \gevc}\\
 & \multicolumn{6}{c}{6 -- 8} & \multicolumn{6}{c}{8 -- 10}\\
\cmidrule(r){2-7}\cmidrule(r){8-13}
 & \multicolumn{3}{c}{\sqs = 7\tev} & \multicolumn{3}{c}{\sqs = 8\tev} & \multicolumn{3}{c}{\sqs = 7\tev} & \multicolumn{3}{c}{\sqs = 8\tev}\\
\cmidrule(r){2-4}\cmidrule(r){5-7}\cmidrule(r){8-10}\cmidrule(r){11-13}
 & \multicolumn{1}{c}{$N_{\chi_{b}(1P)}$} & \multicolumn{1}{c}{$N_{\chi_{b}(2P)}$} & \multicolumn{1}{c}{$N_{\chi_{b}(3P)}$} & \multicolumn{1}{c}{$N_{\chi_{b}(1P)}$} & \multicolumn{1}{c}{$N_{\chi_{b}(2P)}$} & \multicolumn{1}{c}{$N_{\chi_{b}(3P)}$} & \multicolumn{1}{c}{$N_{\chi_{b}(1P)}$} & \multicolumn{1}{c}{$N_{\chi_{b}(2P)}$} & \multicolumn{1}{c}{$N_{\chi_{b}(3P)}$} & \multicolumn{1}{c}{$N_{\chi_{b}(1P)}$} & \multicolumn{1}{c}{$N_{\chi_{b}(2P)}$} & \multicolumn{1}{c}{$N_{\chi_{b}(3P)}$}\\
\midrule
$\lambda=0.0$ & 13.3 & -8.3 & --- & 10.1 & -12.9 & --- & 19.3 & -8.7 & --- & 15.9 & -7.3 & ---\\
$\lambda=0.1$ & 9.0 & -6.7 & --- & 6.1 & -8.9 & --- & 13.9 & -7.0 & --- & 11.1 & -5.9 & ---\\
$\lambda=0.2$ & 5.4 & -5.2 & --- & 2.9 & -5.4 & --- & 9.1 & -5.2 & --- & 6.9 & -4.2 & ---\\

\rule{0pt}{4ex}$\lambda=0.3$ & 2.5 & -3.4 & --- & 0.8 & -3.9 & --- & 4.9 & -3.2 & --- & 3.6 & -2.6 & ---\\
$\lambda=0.4$ & 0.7 & -1.6 & --- & -0.3 & -1.5 & --- & 1.9 & -1.5 & --- & 1.2 & -1.1 & ---\\
$\lambda=0.5$ & 0.0 & 0.0 & --- & 0.0 & 0.0 & --- & 0.0 & 0.0 & --- & 0.0 & 0.0 & ---\\
$\lambda=0.6$ & 0.8 & 1.5 & --- & 1.3 & 0.7 & --- & -0.7 & 1.2 & --- & 0.1 & 0.8 & ---\\
$\lambda=0.7$ & 2.7 & 1.6 & --- & 4.1 & -2.2 & --- & -0.1 & 1.9 & --- & 1.4 & 1.0 & ---\\

\rule{0pt}{4ex}$\lambda=0.8$ & 6.1 & 0.8 & --- & 7.1 & -8.3 & --- & 1.6 & 2.2 & --- & 3.7 & 0.9 & ---\\
$\lambda=0.9$ & 9.4 & -1.3 & --- & 10.9 & -17.2 & --- & 4.3 & 2.0 & --- & 7.1 & 0.5 & ---\\
$\lambda=1.0$ & 13.9 & -5.8 & --- & 14.7 & -24.4 & --- & 7.9 & 1.7 & --- & 11.2 & -0.4 & ---\\
\bottomrule
\end{tabular}
} % scalebox

} % subtable
\subtable[$10 < p_T^{\Y1S} < 18 \gevc$] {
\scalebox{0.65}{
\begin{tabular}{lrrrrrrrrrrrr}\toprule
 & \multicolumn{12}{c}{$\Upsilon(1S)$ transverse momentum intervals, \gevc}\\
 & \multicolumn{6}{c}{10 -- 14} & \multicolumn{6}{c}{14 -- 18}\\
\cmidrule(r){2-7}\cmidrule(r){8-13}
 & \multicolumn{3}{c}{\sqs = 7\tev} & \multicolumn{3}{c}{\sqs = 8\tev} & \multicolumn{3}{c}{\sqs = 7\tev} & \multicolumn{3}{c}{\sqs = 8\tev}\\
\cmidrule(r){2-4}\cmidrule(r){5-7}\cmidrule(r){8-10}\cmidrule(r){11-13}
 & \multicolumn{1}{c}{$N_{\chi_{b}(1P)}$} & \multicolumn{1}{c}{$N_{\chi_{b}(2P)}$} & \multicolumn{1}{c}{$N_{\chi_{b}(3P)}$} & \multicolumn{1}{c}{$N_{\chi_{b}(1P)}$} & \multicolumn{1}{c}{$N_{\chi_{b}(2P)}$} & \multicolumn{1}{c}{$N_{\chi_{b}(3P)}$} & \multicolumn{1}{c}{$N_{\chi_{b}(1P)}$} & \multicolumn{1}{c}{$N_{\chi_{b}(2P)}$} & \multicolumn{1}{c}{$N_{\chi_{b}(3P)}$} & \multicolumn{1}{c}{$N_{\chi_{b}(1P)}$} & \multicolumn{1}{c}{$N_{\chi_{b}(2P)}$} & \multicolumn{1}{c}{$N_{\chi_{b}(3P)}$}\\
\midrule
$\lambda=0.0$ & 15.8 & -3.1 & 36.1 & 13.7 & -13.6 & 36.2 & 6.8 & -0.9 & 5.3 & 7.3 & -2.0 & -9.5\\
$\lambda=0.1$ & 10.7 & -2.6 & 25.9 & 5.5 & -7.0 & -0.6 & 4.2 & -1.0 & 5.5 & 4.7 & -1.9 & -5.7\\
$\lambda=0.2$ & 6.3 & -1.7 & 16.6 & 5.2 & -6.3 & 16.8 & 2.2 & -1.0 & 5.1 & 2.5 & -1.6 & -3.1\\

\rule{0pt}{4ex}$\lambda=0.3$ & 3.0 & -0.9 & 9.0 & 2.6 & -3.4 & 9.3 & 0.9 & -0.9 & 3.9 & 1.1 & -1.2 & -1.2\\
$\lambda=0.4$ & 0.9 & -0.2 & 3.6 & 0.2 & -0.9 & 0.9 & 0.1 & -0.5 & 2.1 & 0.2 & -0.6 & -0.0\\
$\lambda=0.5$ & 0.0 & 0.0 & 0.0 & 0.0 & 0.0 & 0.0 & 0.0 & 0.0 & 0.0 & 0.0 & 0.0 & 0.0\\
$\lambda=0.6$ & 0.4 & -0.2 & -2.1 & 0.6 & 0.6 & -1.5 & 0.7 & 0.3 & -6.2 & 0.5 & 0.9 & -0.7\\
$\lambda=0.7$ & 1.9 & -1.0 & -2.9 & 2.5 & 0.3 & 1.4 & 1.6 & 2.2 & -4.8 & 1.6 & 2.0 & -1.8\\

\rule{0pt}{4ex}$\lambda=0.8$ & 4.4 & -2.0 & -3.0 & 4.1 & -0.2 & -4.8 & 3.1 & 4.4 & -3.5 & 3.2 & 3.4 & -3.4\\
$\lambda=0.9$ & 7.8 & -3.3 & -1.8 & 9.0 & -1.6 & 7.5 & 5.4 & 5.7 & -9.6 & 5.5 & 5.2 & -4.7\\
$\lambda=1.0$ & 12.0 & -4.8 & 0.4 & 13.3 & -3.1 & 9.4 & 8.1 & 8.1 & -11.4 & 8.3 & 7.3 & -5.9\\
\bottomrule
\end{tabular}
} % scalebox

} % subtable
\subtable[$18 < p_T^{\Y1S} < 40 \gevc$] {
\scalebox{0.65}{
\begin{tabular}{lrrrrrrrrrrrr}\toprule
 & \multicolumn{12}{c}{$\Upsilon(1S)$ transverse momentum intervals, \gevc}\\
 & \multicolumn{6}{c}{18 -- 22} & \multicolumn{6}{c}{22 -- 40}\\
\cmidrule(r){2-7}\cmidrule(r){8-13}
 & \multicolumn{3}{c}{\sqs = 7\tev} & \multicolumn{3}{c}{\sqs = 8\tev} & \multicolumn{3}{c}{\sqs = 7\tev} & \multicolumn{3}{c}{\sqs = 8\tev}\\
\cmidrule(r){2-4}\cmidrule(r){5-7}\cmidrule(r){8-10}\cmidrule(r){11-13}
 & \multicolumn{1}{c}{$N_{\chi_{b}(1P)}$} & \multicolumn{1}{c}{$N_{\chi_{b}(2P)}$} & \multicolumn{1}{c}{$N_{\chi_{b}(3P)}$} & \multicolumn{1}{c}{$N_{\chi_{b}(1P)}$} & \multicolumn{1}{c}{$N_{\chi_{b}(2P)}$} & \multicolumn{1}{c}{$N_{\chi_{b}(3P)}$} & \multicolumn{1}{c}{$N_{\chi_{b}(1P)}$} & \multicolumn{1}{c}{$N_{\chi_{b}(2P)}$} & \multicolumn{1}{c}{$N_{\chi_{b}(3P)}$} & \multicolumn{1}{c}{$N_{\chi_{b}(1P)}$} & \multicolumn{1}{c}{$N_{\chi_{b}(2P)}$} & \multicolumn{1}{c}{$N_{\chi_{b}(3P)}$}\\
\midrule
$\lambda=0.0$ & 6.9 & -0.6 & -12.9 & 5.4 & -3.3 & -1.9 & 3.9 & -0.1 & 0.1 & 3.2 & -4.9 & -8.4\\
$\lambda=0.1$ & 4.2 & -0.9 & -9.7 & 3.0 & -3.0 & 0.0 & 1.8 & -0.3 & 0.9 & 0.9 & -3.5 & -6.0\\
$\lambda=0.2$ & 2.2 & -1.0 & -6.9 & 1.2 & -2.5 & 1.1 & 0.4 & -0.2 & 1.6 & -0.6 & -2.2 & -4.0\\

\rule{0pt}{4ex}$\lambda=0.3$ & 0.8 & -0.9 & -4.3 & 0.1 & -1.8 & 1.5 & -0.4 & 0.2 & 2.1 & -1.3 & -1.1 & -2.4\\
$\lambda=0.4$ & 0.0 & -0.5 & -2.1 & -0.3 & -1.0 & 1.2 & -0.6 & 0.8 & 2.6 & -1.1 & -0.4 & -1.0\\
$\lambda=0.5$ & 0.0 & 0.0 & 0.0 & 0.0 & 0.0 & 0.0 & 0.0 & 0.0 & 0.0 & 0.0 & 0.0 & 0.0\\
$\lambda=0.6$ & 0.6 & 0.8 & 2.2 & 0.9 & 1.3 & -1.3 & 0.6 & 2.6 & 3.3 & 1.9 & 0.1 & 0.5\\
$\lambda=0.7$ & 1.8 & 1.8 & 3.9 & 2.4 & 3.0 & -2.7 & 1.9 & 4.0 & 3.8 & 4.3 & 0.2 & 0.6\\

\rule{0pt}{4ex}$\lambda=0.8$ & 3.8 & 3.1 & 5.8 & 4.5 & 5.4 & -3.5 & 3.8 & 5.8 & 4.3 & 7.2 & 0.8 & 0.4\\
$\lambda=0.9$ & 6.3 & 4.9 & 8.5 & 7.2 & 8.0 & -4.5 & 6.3 & 7.9 & 5.1 & 10.5 & 1.9 & 0.6\\
$\lambda=1.0$ & 9.5 & 6.5 & 9.5 & 10.5 & 11.0 & -5.3 & 9.4 & 10.4 & 6.1 & 14.3 & 3.8 & 1.3\\
\bottomrule
\end{tabular}
} % scalebox

} % subtable
\label{tab:syst:lambda_ups1s}
\end{table}

\begin{table}[H]
\centering
\caption{\small $\chi_b$ yields systematic uncertainties (\%) related to \chiboneOneP mass uncertainty in the fit model for $\chi_b(1,2,3P) \to \OneS \gamma$ decays}
\subtable[$6 < p_T^{\Y1S} < 10 \gevc$] {
\scalebox{0.6}{
\begin{tabular}{lrrrrrrrrrrrr}\toprule
 & \multicolumn{12}{c}{$\Upsilon(1S)$ transverse momentum intervals, \gevc}\\
 & \multicolumn{6}{c}{6 -- 8} & \multicolumn{6}{c}{8 -- 10}\\
\cmidrule(r){2-7}\cmidrule(r){8-13}
 & \multicolumn{3}{c}{\sqs = 7\tev} & \multicolumn{3}{c}{\sqs = 8\tev} & \multicolumn{3}{c}{\sqs = 7\tev} & \multicolumn{3}{c}{\sqs = 8\tev}\\
\cmidrule(r){2-4}\cmidrule(r){5-7}\cmidrule(r){8-10}\cmidrule(r){11-13}
 & \multicolumn{1}{c}{$N_{\chi_{b}(1P)}$} & \multicolumn{1}{c}{$N_{\chi_{b}(2P)}$} & \multicolumn{1}{c}{$N_{\chi_{b}(3P)}$} & \multicolumn{1}{c}{$N_{\chi_{b}(1P)}$} & \multicolumn{1}{c}{$N_{\chi_{b}(2P)}$} & \multicolumn{1}{c}{$N_{\chi_{b}(3P)}$} & \multicolumn{1}{c}{$N_{\chi_{b}(1P)}$} & \multicolumn{1}{c}{$N_{\chi_{b}(2P)}$} & \multicolumn{1}{c}{$N_{\chi_{b}(3P)}$} & \multicolumn{1}{c}{$N_{\chi_{b}(1P)}$} & \multicolumn{1}{c}{$N_{\chi_{b}(2P)}$} & \multicolumn{1}{c}{$N_{\chi_{b}(3P)}$}\\
\midrule
Maximum uncertainty & -0.1 & 1.7 & --- & 0.4 & -0.4 & --- & -2.7 & 3.3 & --- & -1.0 & 1.3 & ---\\
\bottomrule
\end{tabular}
} % scalebox

} % subtable
\subtable[$10 < p_T^{\Y1S} < 18 \gevc$] {
\scalebox{0.6}{
\begin{tabular}{lrrrrrrrrrrrr}\toprule
 & \multicolumn{12}{c}{$\Upsilon(1S)$ transverse momentum intervals, \gevc}\\
 & \multicolumn{6}{c}{10 -- 14} & \multicolumn{6}{c}{14 -- 18}\\
\cmidrule(r){2-7}\cmidrule(r){8-13}
 & \multicolumn{3}{c}{\sqs = 7\tev} & \multicolumn{3}{c}{\sqs = 8\tev} & \multicolumn{3}{c}{\sqs = 7\tev} & \multicolumn{3}{c}{\sqs = 8\tev}\\
\cmidrule(r){2-4}\cmidrule(r){5-7}\cmidrule(r){8-10}\cmidrule(r){11-13}
 & \multicolumn{1}{c}{$N_{\chi_{b}(1P)}$} & \multicolumn{1}{c}{$N_{\chi_{b}(2P)}$} & \multicolumn{1}{c}{$N_{\chi_{b}(3P)}$} & \multicolumn{1}{c}{$N_{\chi_{b}(1P)}$} & \multicolumn{1}{c}{$N_{\chi_{b}(2P)}$} & \multicolumn{1}{c}{$N_{\chi_{b}(3P)}$} & \multicolumn{1}{c}{$N_{\chi_{b}(1P)}$} & \multicolumn{1}{c}{$N_{\chi_{b}(2P)}$} & \multicolumn{1}{c}{$N_{\chi_{b}(3P)}$} & \multicolumn{1}{c}{$N_{\chi_{b}(1P)}$} & \multicolumn{1}{c}{$N_{\chi_{b}(2P)}$} & \multicolumn{1}{c}{$N_{\chi_{b}(3P)}$}\\
\midrule
Maximum uncertainty & -0.4 & -0.6 & -3.7 & 0.1 & -0.3 & 0.5 & 0.5 & -0.7 & 3.4 & 0.5 & -0.8 & -1.2\\
\bottomrule
\end{tabular}
} % scalebox

} % subtable
\subtable[$18 < p_T^{\Y1S} < 40 \gevc$] {
\scalebox{0.6}{
\begin{tabular}{lrrrrrrrrrrrr}\toprule
 & \multicolumn{12}{c}{$\Upsilon(1S)$ transverse momentum intervals, \gevc}\\
 & \multicolumn{6}{c}{18 -- 22} & \multicolumn{6}{c}{22 -- 40}\\
\cmidrule(r){2-7}\cmidrule(r){8-13}
 & \multicolumn{3}{c}{\sqs = 7\tev} & \multicolumn{3}{c}{\sqs = 8\tev} & \multicolumn{3}{c}{\sqs = 7\tev} & \multicolumn{3}{c}{\sqs = 8\tev}\\
\cmidrule(r){2-4}\cmidrule(r){5-7}\cmidrule(r){8-10}\cmidrule(r){11-13}
 & \multicolumn{1}{c}{$N_{\chi_{b}(1P)}$} & \multicolumn{1}{c}{$N_{\chi_{b}(2P)}$} & \multicolumn{1}{c}{$N_{\chi_{b}(3P)}$} & \multicolumn{1}{c}{$N_{\chi_{b}(1P)}$} & \multicolumn{1}{c}{$N_{\chi_{b}(2P)}$} & \multicolumn{1}{c}{$N_{\chi_{b}(3P)}$} & \multicolumn{1}{c}{$N_{\chi_{b}(1P)}$} & \multicolumn{1}{c}{$N_{\chi_{b}(2P)}$} & \multicolumn{1}{c}{$N_{\chi_{b}(3P)}$} & \multicolumn{1}{c}{$N_{\chi_{b}(1P)}$} & \multicolumn{1}{c}{$N_{\chi_{b}(2P)}$} & \multicolumn{1}{c}{$N_{\chi_{b}(3P)}$}\\
\midrule
Maximum uncertainty & 0.9 & -0.5 & -7.0 & 0.4 & -2.3 & 2.3 & 0.6 & -1.9 & -2.2 & -0.5 & -4.1 & -10.8\\
\bottomrule
\end{tabular}
} % scalebox

} % subtable
\label{tab:syst:m_ups1s}
\end{table}

\begin{table}[H]
\centering
\caption{\small $\chi_b$ yields systematic uncertainties (\%) related to \chiboneThreeP mass uncertainty in the fit model for $\chi_b(1,2,3P) \to \OneS \gamma$ decays}
\subtable[$6 < p_T^{\Y1S} < 10 \gevc$] {
\scalebox{0.6}{
\begin{tabular}{lrrrrrrrrrrrr}\toprule
 & \multicolumn{12}{c}{$\Upsilon(1S)$ transverse momentum intervals, \gevc}\\
 & \multicolumn{6}{c}{6 -- 8} & \multicolumn{6}{c}{8 -- 10}\\
\cmidrule(r){2-7}\cmidrule(r){8-13}
 & \multicolumn{3}{c}{\sqs = 7\tev} & \multicolumn{3}{c}{\sqs = 8\tev} & \multicolumn{3}{c}{\sqs = 7\tev} & \multicolumn{3}{c}{\sqs = 8\tev}\\
\cmidrule(r){2-4}\cmidrule(r){5-7}\cmidrule(r){8-10}\cmidrule(r){11-13}
 & \multicolumn{1}{c}{$N_{\chi_{b}(1P)}$} & \multicolumn{1}{c}{$N_{\chi_{b}(2P)}$} & \multicolumn{1}{c}{$N_{\chi_{b}(3P)}$} & \multicolumn{1}{c}{$N_{\chi_{b}(1P)}$} & \multicolumn{1}{c}{$N_{\chi_{b}(2P)}$} & \multicolumn{1}{c}{$N_{\chi_{b}(3P)}$} & \multicolumn{1}{c}{$N_{\chi_{b}(1P)}$} & \multicolumn{1}{c}{$N_{\chi_{b}(2P)}$} & \multicolumn{1}{c}{$N_{\chi_{b}(3P)}$} & \multicolumn{1}{c}{$N_{\chi_{b}(1P)}$} & \multicolumn{1}{c}{$N_{\chi_{b}(2P)}$} & \multicolumn{1}{c}{$N_{\chi_{b}(3P)}$}\\
\midrule
$m_{\chiboneThreeP} = 10,502 \mevcc$ & 0.0 & -0.2 & --- & -0.0 & -0.2 & --- & 0.0 & 0.0 & --- & 0.0 & 0.0 & ---\\
$m_{\chiboneThreeP} = 10,518 \mevcc$ & 0.0 & -0.2 & --- & -0.0 & -0.2 & --- & 0.0 & 0.0 & --- & 0.0 & 0.0 & ---\\
\bottomrule
\end{tabular}
} % scalebox

} % subtable
\subtable[$10 < p_T^{\Y1S} < 18 \gevc$] {
\scalebox{0.6}{
\begin{tabular}{lrrrrrrrrrrrr}\toprule
 & \multicolumn{12}{c}{$\Upsilon(1S)$ transverse momentum intervals, \gevc}\\
 & \multicolumn{6}{c}{10 -- 14} & \multicolumn{6}{c}{14 -- 18}\\
\cmidrule(r){2-7}\cmidrule(r){8-13}
 & \multicolumn{3}{c}{\sqs = 7\tev} & \multicolumn{3}{c}{\sqs = 8\tev} & \multicolumn{3}{c}{\sqs = 7\tev} & \multicolumn{3}{c}{\sqs = 8\tev}\\
\cmidrule(r){2-4}\cmidrule(r){5-7}\cmidrule(r){8-10}\cmidrule(r){11-13}
 & \multicolumn{1}{c}{$N_{\chi_{b}(1P)}$} & \multicolumn{1}{c}{$N_{\chi_{b}(2P)}$} & \multicolumn{1}{c}{$N_{\chi_{b}(3P)}$} & \multicolumn{1}{c}{$N_{\chi_{b}(1P)}$} & \multicolumn{1}{c}{$N_{\chi_{b}(2P)}$} & \multicolumn{1}{c}{$N_{\chi_{b}(3P)}$} & \multicolumn{1}{c}{$N_{\chi_{b}(1P)}$} & \multicolumn{1}{c}{$N_{\chi_{b}(2P)}$} & \multicolumn{1}{c}{$N_{\chi_{b}(3P)}$} & \multicolumn{1}{c}{$N_{\chi_{b}(1P)}$} & \multicolumn{1}{c}{$N_{\chi_{b}(2P)}$} & \multicolumn{1}{c}{$N_{\chi_{b}(3P)}$}\\
\midrule
$m_{\chiboneThreeP} = 10,502 \mevcc$ & 0.2 & -0.8 & -4.1 & -0.2 & -0.6 & -2.5 & 0.1 & -0.9 & -5.8 & -0.1 & 0.3 & 4.6\\
$m_{\chiboneThreeP} = 10,518 \mevcc$ & -0.1 & 1.7 & 12.7 & -0.4 & 1.8 & 0.9 & -0.3 & 2.5 & 15.0 & 0.1 & -0.3 & -5.4\\
\bottomrule
\end{tabular}
} % scalebox

} % subtable
\subtable[$18 < p_T^{\Y1S} < 40 \gevc$] {
\scalebox{0.6}{
\begin{tabular}{lrrrrrrrrrrrr}\toprule
 & \multicolumn{12}{c}{$\Upsilon(1S)$ transverse momentum intervals, \gevc}\\
 & \multicolumn{6}{c}{18 -- 22} & \multicolumn{6}{c}{22 -- 40}\\
\cmidrule(r){2-7}\cmidrule(r){8-13}
 & \multicolumn{3}{c}{\sqs = 7\tev} & \multicolumn{3}{c}{\sqs = 8\tev} & \multicolumn{3}{c}{\sqs = 7\tev} & \multicolumn{3}{c}{\sqs = 8\tev}\\
\cmidrule(r){2-4}\cmidrule(r){5-7}\cmidrule(r){8-10}\cmidrule(r){11-13}
 & \multicolumn{1}{c}{$N_{\chi_{b}(1P)}$} & \multicolumn{1}{c}{$N_{\chi_{b}(2P)}$} & \multicolumn{1}{c}{$N_{\chi_{b}(3P)}$} & \multicolumn{1}{c}{$N_{\chi_{b}(1P)}$} & \multicolumn{1}{c}{$N_{\chi_{b}(2P)}$} & \multicolumn{1}{c}{$N_{\chi_{b}(3P)}$} & \multicolumn{1}{c}{$N_{\chi_{b}(1P)}$} & \multicolumn{1}{c}{$N_{\chi_{b}(2P)}$} & \multicolumn{1}{c}{$N_{\chi_{b}(3P)}$} & \multicolumn{1}{c}{$N_{\chi_{b}(1P)}$} & \multicolumn{1}{c}{$N_{\chi_{b}(2P)}$} & \multicolumn{1}{c}{$N_{\chi_{b}(3P)}$}\\
\midrule
$m_{\chiboneThreeP} = 10,502 \mevcc$ & -0.2 & 1.4 & 13.3 & 0.0 & -0.5 & -3.5 & -0.1 & 0.2 & 2.9 & -0.1 & 0.8 & 7.5\\
$m_{\chiboneThreeP} = 10,518 \mevcc$ & 0.3 & -1.9 & -17.6 & -0.1 & 0.8 & 7.7 & -0.1 & 1.2 & -0.4 & -0.0 & -0.7 & -10.0\\
\bottomrule
\end{tabular}
} % scalebox

} % subtable
\label{tab:syst:m3p_ups1s}
\end{table}

\begin{table}[H]
\centering
\caption{\small $\chi_b$ yields systematic uncertainties (\%) related to $\lambda$ values in the fit model for  $\chi_b(2,3P) \to \TwoS \gamma$ decays}
\subtable[$18 < p_T^{\Y2S} < 24 \gevc$] {
\scalebox{0.65}{
\begin{tabular}{lrrrrrrrrrrrr}\toprule
 & \multicolumn{12}{c}{$\Upsilon(2S)$ transverse momentum intervals, \gevc}\\
 & \multicolumn{4}{c}{18 -- 22} & \multicolumn{4}{c}{18 -- 24} & \multicolumn{4}{c}{22 -- 24}\\
\cmidrule(r){2-5}\cmidrule(r){6-9}\cmidrule(r){10-13}
 & \multicolumn{2}{c}{\sqs = 7\tev} & \multicolumn{2}{c}{\sqs = 8\tev} & \multicolumn{2}{c}{\sqs = 7\tev} & \multicolumn{2}{c}{\sqs = 8\tev} & \multicolumn{2}{c}{\sqs = 7\tev} & \multicolumn{2}{c}{\sqs = 8\tev}\\
\cmidrule(r){2-3}\cmidrule(r){4-5}\cmidrule(r){6-7}\cmidrule(r){8-9}\cmidrule(r){10-11}\cmidrule(r){12-13}
 & \multicolumn{1}{c}{$N_{\chi_{b}(2P)}$} & \multicolumn{1}{c}{$N_{\chi_{b}(3P)}$} & \multicolumn{1}{c}{$N_{\chi_{b}(2P)}$} & \multicolumn{1}{c}{$N_{\chi_{b}(3P)}$} & \multicolumn{1}{c}{$N_{\chi_{b}(2P)}$} & \multicolumn{1}{c}{$N_{\chi_{b}(3P)}$} & \multicolumn{1}{c}{$N_{\chi_{b}(2P)}$} & \multicolumn{1}{c}{$N_{\chi_{b}(3P)}$} & \multicolumn{1}{c}{$N_{\chi_{b}(2P)}$} & \multicolumn{1}{c}{$N_{\chi_{b}(3P)}$} & \multicolumn{1}{c}{$N_{\chi_{b}(2P)}$} & \multicolumn{1}{c}{$N_{\chi_{b}(3P)}$}\\
\midrule
$\lambda=0.0$ & 22.1 & --- & 13.6 & --- & --- & 18.8 & --- & 12.6 & 5.6 & --- & 3.8 & ---\\
$\lambda=0.1$ & 15.8 & --- & 9.3 & --- & --- & 14.0 & --- & 8.8 & 2.3 & --- & 1.1 & ---\\
$\lambda=0.2$ & 10.1 & --- & 5.8 & --- & --- & 9.2 & --- & 5.6 & -0.1 & --- & -0.7 & ---\\

\rule{0pt}{4ex}$\lambda=0.3$ & 5.6 & --- & 3.2 & --- & --- & 5.4 & --- & 3.0 & -1.3 & --- & -1.6 & ---\\
$\lambda=0.4$ & 2.1 & --- & 1.1 & --- & --- & 2.2 & --- & 1.0 & -1.3 & --- & -1.3 & ---\\
$\lambda=0.5$ & 0.0 & --- & 0.0 & --- & --- & 0.0 & --- & 0.0 & 0.0 & --- & 0.0 & ---\\
$\lambda=0.6$ & -1.0 & --- & -0.0 & --- & --- & -1.1 & --- & 1.3 & 2.8 & --- & 2.1 & ---\\
$\lambda=0.7$ & -0.8 & --- & 0.9 & --- & --- & -1.3 & --- & 3.5 & 6.4 & --- & 5.6 & ---\\

\rule{0pt}{4ex}$\lambda=0.8$ & 0.5 & --- & 2.9 & --- & --- & -0.2 & --- & 5.4 & 11.5 & --- & 10.0 & ---\\
$\lambda=0.9$ & 2.8 & --- & 5.8 & --- & --- & 1.6 & --- & 7.2 & 17.5 & --- & 15.3 & ---\\
$\lambda=1.0$ & 6.2 & --- & 9.1 & --- & --- & 4.1 & --- & 10.5 & 24.3 & --- & 21.3 & ---\\
\bottomrule
\end{tabular}
} % scalebox

} % subtable
\subtable[$24 < p_T^{\Y2S} < 40 \gevc$] {
\scalebox{0.65}{
\begin{tabular}{lrrrr}\toprule
 & \multicolumn{4}{c}{$\Upsilon(2S)$ transverse momentum intervals, \gevc}\\
 & \multicolumn{4}{c}{24 -- 40}\\
\cmidrule(r){2-5}
 & \multicolumn{2}{c}{\sqs = 7\tev} & \multicolumn{2}{c}{\sqs = 8\tev}\\
\cmidrule(r){2-3}\cmidrule(r){4-5}
 & \multicolumn{1}{c}{$N_{\chi_{b}(2P)}$} & \multicolumn{1}{c}{$N_{\chi_{b}(3P)}$} & \multicolumn{1}{c}{$N_{\chi_{b}(2P)}$} & \multicolumn{1}{c}{$N_{\chi_{b}(3P)}$}\\
\midrule
$\lambda=0.0$ & 12.5 & 3.7 & 13.8 & -9.6\\
$\lambda=0.1$ & 8.1 & 2.1 & 9.5 & -9.6\\
$\lambda=0.2$ & 4.6 & 1.0 & 5.9 & -8.5\\

\rule{0pt}{4ex}$\lambda=0.3$ & 2.0 & 0.4 & 3.1 & -6.6\\
$\lambda=0.4$ & 0.5 & -0.1 & 1.1 & -3.9\\
$\lambda=0.5$ & 0.0 & 0.0 & 0.0 & 0.0\\
$\lambda=0.6$ & 0.9 & 0.0 & -0.3 & 4.3\\
$\lambda=0.7$ & 2.8 & 0.5 & 0.3 & 8.9\\

\rule{0pt}{4ex}$\lambda=0.8$ & 5.7 & 1.1 & 1.7 & 14.3\\
$\lambda=0.9$ & 9.6 & 2.3 & 3.9 & 20.1\\
$\lambda=1.0$ & 14.4 & 3.9 & 6.8 & 25.8\\
\bottomrule
\end{tabular}
} % scalebox

} % subtable
\label{tab:syst:lambda_ups2s}
\end{table}

\begin{table}[H]
\centering
\caption{\small $\chi_b$ yields systematic uncertainties (\%) related to \chiboneTwoP mass uncertainty in the fit model for $\chi_b(2,3P) \to \TwoS \gamma$ decays}
\subtable[$18 < p_T^{\Y2S} < 24 \gevc$] {
\scalebox{0.6}{
\begin{tabular}{lrrrrrrrrrrrr}\toprule
 & \multicolumn{12}{c}{$\Upsilon(2S)$ transverse momentum intervals, \gevc}\\
 & \multicolumn{4}{c}{18 -- 22} & \multicolumn{4}{c}{18 -- 24} & \multicolumn{4}{c}{22 -- 24}\\
\cmidrule(r){2-5}\cmidrule(r){6-9}\cmidrule(r){10-13}
 & \multicolumn{2}{c}{\sqs = 7\tev} & \multicolumn{2}{c}{\sqs = 8\tev} & \multicolumn{2}{c}{\sqs = 7\tev} & \multicolumn{2}{c}{\sqs = 8\tev} & \multicolumn{2}{c}{\sqs = 7\tev} & \multicolumn{2}{c}{\sqs = 8\tev}\\
\cmidrule(r){2-3}\cmidrule(r){4-5}\cmidrule(r){6-7}\cmidrule(r){8-9}\cmidrule(r){10-11}\cmidrule(r){12-13}
 & \multicolumn{1}{c}{$N_{\chi_{b}(2P)}$} & \multicolumn{1}{c}{$N_{\chi_{b}(3P)}$} & \multicolumn{1}{c}{$N_{\chi_{b}(2P)}$} & \multicolumn{1}{c}{$N_{\chi_{b}(3P)}$} & \multicolumn{1}{c}{$N_{\chi_{b}(2P)}$} & \multicolumn{1}{c}{$N_{\chi_{b}(3P)}$} & \multicolumn{1}{c}{$N_{\chi_{b}(2P)}$} & \multicolumn{1}{c}{$N_{\chi_{b}(3P)}$} & \multicolumn{1}{c}{$N_{\chi_{b}(2P)}$} & \multicolumn{1}{c}{$N_{\chi_{b}(3P)}$} & \multicolumn{1}{c}{$N_{\chi_{b}(2P)}$} & \multicolumn{1}{c}{$N_{\chi_{b}(3P)}$}\\
\midrule
Maximum uncertainty & -1.1 & --- & -0.4 & --- & --- & -2.6 & --- & 1.7 & -1.4 & --- & -0.2 & ---\\
\bottomrule
\end{tabular}
} % scalebox

} % subtable
\subtable[$24 < p_T^{\Y2S} < 40 \gevc$] {
\scalebox{0.6}{
\begin{tabular}{lrrrr}\toprule
 & \multicolumn{4}{c}{$\Upsilon(2S)$ transverse momentum intervals, \gevc}\\
 & \multicolumn{4}{c}{24 -- 40}\\
\cmidrule(r){2-5}
 & \multicolumn{2}{c}{\sqs = 7\tev} & \multicolumn{2}{c}{\sqs = 8\tev}\\
\cmidrule(r){2-3}\cmidrule(r){4-5}
 & \multicolumn{1}{c}{$N_{\chi_{b}(2P)}$} & \multicolumn{1}{c}{$N_{\chi_{b}(3P)}$} & \multicolumn{1}{c}{$N_{\chi_{b}(2P)}$} & \multicolumn{1}{c}{$N_{\chi_{b}(3P)}$}\\
\midrule
Maximum uncertainty & 2.8 & 1.3 & -0.5 & 10.5\\
\bottomrule
\end{tabular}
} % scalebox

} % subtable
\label{tab:syst:m_ups2s}
\end{table}

\begin{table}[H]
\centering
\caption{\small $\chi_b$ yields systematic uncertainties (\%) related to \chiboneThreeP mass uncertainty in the fit model for  $\chi_b(2,3P) \to \TwoS \gamma$ decays}
\subtable[$18 < p_T^{\Y2S} < 24 \gevc$] {
\scalebox{0.6}{
\begin{tabular}{lrrrrrrrrrrrr}\toprule
 & \multicolumn{12}{c}{$\Upsilon(2S)$ transverse momentum intervals, \gevc}\\
 & \multicolumn{4}{c}{18 -- 22} & \multicolumn{4}{c}{18 -- 24} & \multicolumn{4}{c}{22 -- 24}\\
\cmidrule(r){2-5}\cmidrule(r){6-9}\cmidrule(r){10-13}
 & \multicolumn{2}{c}{\sqs = 7\tev} & \multicolumn{2}{c}{\sqs = 8\tev} & \multicolumn{2}{c}{\sqs = 7\tev} & \multicolumn{2}{c}{\sqs = 8\tev} & \multicolumn{2}{c}{\sqs = 7\tev} & \multicolumn{2}{c}{\sqs = 8\tev}\\
\cmidrule(r){2-3}\cmidrule(r){4-5}\cmidrule(r){6-7}\cmidrule(r){8-9}\cmidrule(r){10-11}\cmidrule(r){12-13}
 & \multicolumn{1}{c}{$N_{\chi_{b}(2P)}$} & \multicolumn{1}{c}{$N_{\chi_{b}(3P)}$} & \multicolumn{1}{c}{$N_{\chi_{b}(2P)}$} & \multicolumn{1}{c}{$N_{\chi_{b}(3P)}$} & \multicolumn{1}{c}{$N_{\chi_{b}(2P)}$} & \multicolumn{1}{c}{$N_{\chi_{b}(3P)}$} & \multicolumn{1}{c}{$N_{\chi_{b}(2P)}$} & \multicolumn{1}{c}{$N_{\chi_{b}(3P)}$} & \multicolumn{1}{c}{$N_{\chi_{b}(2P)}$} & \multicolumn{1}{c}{$N_{\chi_{b}(3P)}$} & \multicolumn{1}{c}{$N_{\chi_{b}(2P)}$} & \multicolumn{1}{c}{$N_{\chi_{b}(3P)}$}\\
\midrule
$m_{\chiboneThreeP} = 10,502 \mevcc$ & -0.0 & --- & -0.1 & --- & --- & -4.5 & --- & 0.3 & -0.3 & --- & -0.1 & ---\\
$m_{\chiboneThreeP} = 10,518 \mevcc$ & -0.0 & --- & 0.1 & --- & --- & 12.0 & --- & 6.4 & 1.0 & --- & -0.2 & ---\\
\bottomrule
\end{tabular}
} % scalebox

} % subtable
\subtable[$24 < p_T^{\Y2S} < 40 \gevc$] {
\scalebox{0.6}{
\begin{tabular}{lrrrr}\toprule
 & \multicolumn{4}{c}{$\Upsilon(2S)$ transverse momentum intervals, \gevc}\\
 & \multicolumn{4}{c}{24 -- 40}\\
\cmidrule(r){2-5}
 & \multicolumn{2}{c}{\sqs = 7\tev} & \multicolumn{2}{c}{\sqs = 8\tev}\\
\cmidrule(r){2-3}\cmidrule(r){4-5}
 & \multicolumn{1}{c}{$N_{\chi_{b}(2P)}$} & \multicolumn{1}{c}{$N_{\chi_{b}(3P)}$} & \multicolumn{1}{c}{$N_{\chi_{b}(2P)}$} & \multicolumn{1}{c}{$N_{\chi_{b}(3P)}$}\\
\midrule
$m_{\chiboneThreeP} = 10,502 \mevcc$ & 0.1 & -0.0 & 0.1 & 13.7\\
$m_{\chiboneThreeP} = 10,518 \mevcc$ & 0.6 & 10.0 & -0.1 & -18.1\\
\bottomrule
\end{tabular}
} % scalebox

} % subtable
\label{tab:syst:m3p_ups2s}
\end{table}

\begin{table}[H]
\caption{\small $\chi_b$ yields systematic uncertainties (\%) related to $\lambda$ values in the fit model for  $\chi_b(3P) \to \ThreeS \gamma$ decays}
\centering
\scalebox{0.65}{
\begin{tabular}{lrr}\toprule
 & \multicolumn{2}{c}{$\Upsilon(3S)$ transverse momentum intervals, \gevc}\\
 & \multicolumn{2}{c}{27 -- 40}\\
\cmidrule(r){2-3}
 & \multicolumn{1}{c}{\sqs = 7\tev} & \multicolumn{1}{c}{\sqs = 8\tev}\\
\cmidrule(r){2-2}\cmidrule(r){3-3}
 & \multicolumn{1}{c}{$N_{\chi_{b}(3P)}$} & \multicolumn{1}{c}{$N_{\chi_{b}(3P)}$}\\
\midrule
$\lambda=0.0$ & -11.3 & 17.8\\
$\lambda=0.1$ & -13.0 & 17.8\\
$\lambda=0.2$ & -12.9 & 6.6\\

\rule{0pt}{4ex}$\lambda=0.3$ & -8.5 & 11.5\\
$\lambda=0.4$ & -5.1 & 0.9\\
$\lambda=0.5$ & 0.0 & 0.0\\
$\lambda=0.6$ & 6.8 & 9.8\\
$\lambda=0.7$ & 15.6 & 1.5\\

\rule{0pt}{4ex}$\lambda=0.8$ & 25.1 & 12.9\\
$\lambda=0.9$ & 35.4 & 7.0\\
$\lambda=1.0$ & 46.6 & 10.8\\
\bottomrule
\end{tabular}
} % scalebox
\label{tab:syst:lambda_ups3s}
\end{table}

\begin{table}[H]
\caption{\small $\chi_b$ yields systematic uncertainties (\%) related to \chiboneThreeP mass uncertainty in the fit model for  $\chi_b(3P) \to \ThreeS \gamma$ decays}
\centering
\scalebox{0.65}{
\begin{tabular}{lrr}\toprule
 & \multicolumn{2}{c}{$\Upsilon(3S)$ transverse momentum intervals, \gevc}\\
 & \multicolumn{2}{c}{27 -- 40}\\
\cmidrule(r){2-3}
 & \multicolumn{1}{c}{\sqs = 7\tev} & \multicolumn{1}{c}{\sqs = 8\tev}\\
\cmidrule(r){2-2}\cmidrule(r){3-3}
 & \multicolumn{1}{c}{$N_{\chi_{b}(3P)}$} & \multicolumn{1}{c}{$N_{\chi_{b}(3P)}$}\\
\midrule
$m_{\chiboneThreeP} = 10,502 \mevcc$ & 23.4 & 3.4\\
$m_{\chiboneThreeP} = 10,518 \mevcc$ & -23.7 & 22.8\\
\bottomrule
\end{tabular}
} % scalebox
\label{tab:syst:m3p_ups3s}
\end{table}


\subsection{Photon reconstruction efficiency}
The photon reconstruction efficiency, taken from simulation, needs not to be
the same as in real data. The detailed comparison between MonteCarlo and data,
presented in Section~\ref{sec:mc:datavsmc}, shows that the differences
are small. We assign a systematic uncertainty based on previous studies of
photon reconstruction efficiencies. These studies compare the
$B^+ \to \jpsi K^{*+}$ and $B^+ \to \jpsi K^+$ yields in data and MonteCarlo in
order to determine the neutral pion, hence the photon  reconstruction
efficiency. A systematic uncertainty of 3\% is assigned to this effect.


\subsection{\texorpdfstring{\chib}{chib} polarization}
\label{sec:syst:pol}

The prompt \chib polarization is unknown. The simulated \chib mesons are
unpolarized and all the efficiencies given in the previous sections are
therefore determined under the assumption that the $\chi_{b1}$ and the
$\chi_{b2}$ mesons are produced unpolarized. The photon and $\Upsilon$ momentum
distributions depend on the polarization of the \chib state and the same is
true for the ratio of efficiencies. The correction factors for the ratio of
efficiencies under other polarization scenarios are derived here.

The angular distribution of the $\chib \to \Upsilon \gamma$ decay is described
by the angles $\theta_{\Upsilon}$ , $\theta_{\chib}$ and $\phi$ where:
$\theta_{\Upsilon}$ is the angle between the directions of the positive muon in
the $\Upsilon$ rest frame and the $\Upsilon$ in the $\chib$ rest frame;
$\theta_{\chib}$ is the angle between the directions of the $\Upsilon$ in the
\chib rest frame and the \chib in the laboratory frame; $\phi$ is the angle
between the $\Upsilon$ decay plane in the \chib rest frame and the plane formed
by the \chib direction in the laboratory frame and the direction of the
$\Upsilon$ in the \chib rest frame. The angular distributions of the
\chib states depend on $m_{\chi_{bJ}}$ , which is the azimuthal angular
momentum quantum number of the $\chi_{bJ}$ state. For each simulated event in
the unpolarized sample, a weight is calculated from the values of described
angles in the various polarization hypotheses and the ratio of efficiencies is
deduced for each ($m_{\chi_{b1}}$, $m_{\chi_{b2}}$) polarization combination.

As an example, \Cref{fig:syst:polarization:angles_chib11p_ups1s,fig:syst:polarization:angles_chib21p_ups1s} show the angular
distributions  in the $\chi_{b1,2}(1P)\to\Upsilon(1S)\gamma$
decay for unpolarized and various polarization scenarios for the $\chi_b$. The
resulting efficiency ratios are shown
in~\Cref{fig:syst:polarization:eratio_chib1p}. The statistical errors are estimated by
jacknife method\cite{jacknife,wouter}. The corresponding ratios for
$\chi_{b1,2}(2P) \to \Upsilon(1S) \gamma$ decays are given in~\Cref{fig:syst:polarization:eratio_chib2p}.


\begin{figure}[H]
  \setlength{\unitlength}{1mm}
  \centering
  \scalebox{0.6} {
  \begin{picture}(225,120)
  	\put(0,0){
      \includegraphics*[width=75mm, height=45mm]{polarization/angles/w1_cosphi_chib11p_ups1s}
    }
    \put(75,0){
      \includegraphics*[width=75mm, height=45mm]{polarization/angles/w1_theta_chib11p_ups1s}
    }
    \put(150,0){
      \includegraphics*[width=75mm, height=45mm]{polarization/angles/w1_thetap_chib11p_ups1s}
    }
	\put(0,60){
      \includegraphics*[width=75mm, height=45mm]{polarization/angles/w0_cosphi_chib11p_ups1s}
    }
    \put(75,60){
      \includegraphics*[width=75mm, height=45mm]{polarization/angles/w0_theta_chib11p_ups1s}
    }
    \put(150,60){
      \includegraphics*[width=75mm, height=45mm]{polarization/angles/w0_thetap_chib11p_ups1s}
    }

    \put(60,0){$\cos(\phi)$}
    \put(60,60){$\cos(\phi)$}

    \put(135,0){$\cos(\theta_{\chib})$}
    \put(135,60){$\cos(\theta_{\chib})$}

    \put(210,0){$\cos(\theta_{\Upsilon})$}
    \put(210,60){$\cos(\theta_{\Upsilon})$}


	\put(2,10){\begin{sideways}Arbitrary units\end{sideways}}
    \put(2,70){\begin{sideways}Arbitrary units\end{sideways}}

    \put(77,10){\begin{sideways}Arbitrary units\end{sideways}}
    \put(77,70){\begin{sideways}Arbitrary units\end{sideways}}

    \put(152,10){\begin{sideways}Arbitrary units\end{sideways}}
    \put(152,70){\begin{sideways}Arbitrary units\end{sideways}}

    \put(45,35){\includegraphics*[width=4mm, height=2mm]{blue}}
    \put(45,32){\includegraphics*[width=4mm, height=2mm]{red}}

    \put(45,95){\includegraphics*[width=4mm, height=2mm]{blue}}
    \put(45,92){\includegraphics*[width=4mm, height=2mm]{red}}

    \put(120,35){\includegraphics*[width=4mm, height=2mm]{blue}}
    \put(120,32){\includegraphics*[width=4mm, height=2mm]{red}}

    \put(120,95){\includegraphics*[width=4mm, height=2mm]{blue}}
    \put(120,92){\includegraphics*[width=4mm, height=2mm]{red}}

    \put(195,15){\includegraphics*[width=4mm, height=2mm]{blue}}
    \put(195,12){\includegraphics*[width=4mm, height=2mm]{red}}

    \put(195,75){\includegraphics*[width=4mm, height=2mm]{blue}}
    \put(195,72){\includegraphics*[width=4mm, height=2mm]{red}}    


    \put(50,35){unpolarized}
    \put(50,32){$|m_{\chi_{b1}}|=1$}

    \put(50,95){unpolarized}
    \put(50,92){$|m_{\chi_{b1}}|=0$}

    \put(125,35){unpolarized}
    \put(125,32){$|m_{\chi_{b1}}|=1$}

    \put(125,95){unpolarized}
    \put(125,92){$|m_{\chi_{b1}}|=0$}

    \put(200,15){unpolarized}
    \put(200,12){$|m_{\chi_{b1}}|=1$}

    \put(200,75){unpolarized}
    \put(200,72){$|m_{\chi_{b1}}|=0$}    

  \end{picture}
  }
\caption {\small
	Angular distributions of simulated events in $\chi_{b1}(1P) \to \Y1S \gamma$
	decay. The blue curves corresponds to unpolarized events distribution and
	the red curves corresponds to specified polarized events distribution. All
	histograms are normalized by the corresponding integral. }
\label{fig:syst:polarization:angles_chib11p_ups1s}
\end{figure}


\begin{figure}[H]
  \setlength{\unitlength}{1mm}
  \centering
  \scalebox{0.6} {
  \begin{picture}(225,180)
  	\put(0,0){
      \includegraphics*[width=75mm, height=45mm]{polarization/angles/w2_cosphi_chib21p_ups1s}
    }
    \put(75,0){
      \includegraphics*[width=75mm, height=45mm]{polarization/angles/w2_theta_chib21p_ups1s}
    }
    \put(150,0){
      \includegraphics*[width=75mm, height=45mm]{polarization/angles/w2_thetap_chib21p_ups1s}
    }
	\put(0,60){
      \includegraphics*[width=75mm, height=45mm]{polarization/angles/w1_cosphi_chib21p_ups1s}
    }
    \put(75,60){
      \includegraphics*[width=75mm, height=45mm]{polarization/angles/w1_theta_chib21p_ups1s}
    }
    \put(150,60){
      \includegraphics*[width=75mm, height=45mm]{polarization/angles/w1_thetap_chib21p_ups1s}
    }
    \put(0,120){
      \includegraphics*[width=75mm, height=45mm]{polarization/angles/w0_cosphi_chib21p_ups1s}
    }
    \put(75,120){
      \includegraphics*[width=75mm, height=45mm]{polarization/angles/w0_theta_chib21p_ups1s}
    }
    \put(150,120){
      \includegraphics*[width=75mm, height=45mm]{polarization/angles/w0_thetap_chib21p_ups1s}
    }

    \put(60,0){$\cos(\phi)$}
    \put(60,60){$\cos(\phi)$}
    \put(60,120){$\cos(\phi)$}

    \put(135,0){$\cos(\theta_{\chib})$}
    \put(135,60){$\cos(\theta_{\chib})$}
    \put(135,120){$\cos(\theta_{\chib})$}

    \put(210,0){$\cos(\theta_{\Upsilon})$}
    \put(210,60){$\cos(\theta_{\Upsilon})$}
    \put(210,120){$\cos(\theta_{\Upsilon})$}


	  \put(2,10){\begin{sideways}Arbitrary units\end{sideways}}
    \put(2,70){\begin{sideways}Arbitrary units\end{sideways}}
    \put(2,130){\begin{sideways}Arbitrary units\end{sideways}}

    \put(77,10){\begin{sideways}Arbitrary units\end{sideways}}
    \put(77,70){\begin{sideways}Arbitrary units\end{sideways}}
    \put(77,130){\begin{sideways}Arbitrary units\end{sideways}}

    \put(152,10){\begin{sideways}Arbitrary units\end{sideways}}
    \put(152,70){\begin{sideways}Arbitrary units\end{sideways}}
    \put(152,130){\begin{sideways}Arbitrary units\end{sideways}}

    \put(40,37){\includegraphics*[width=4mm, height=2mm]{blue}}
    \put(40,34){\includegraphics*[width=4mm, height=2mm]{red}}

    \put(40,97){\includegraphics*[width=4mm, height=2mm]{blue}}
    \put(40,94){\includegraphics*[width=4mm, height=2mm]{red}}

    \put(40,156){\includegraphics*[width=4mm, height=2mm]{blue}}
    \put(40,154){\includegraphics*[width=4mm, height=2mm]{red}}

    \put(115,37){\includegraphics*[width=4mm, height=2mm]{blue}}
    \put(115,34){\includegraphics*[width=4mm, height=2mm]{red}}

    \put(115,97){\includegraphics*[width=4mm, height=2mm]{blue}}
    \put(115,94){\includegraphics*[width=4mm, height=2mm]{red}}

    \put(115,136){\includegraphics*[width=4mm, height=2mm]{blue}}
    \put(115,133){\includegraphics*[width=4mm, height=2mm]{red}}


    \put(190,37){\includegraphics*[width=4mm, height=2mm]{blue}}
    \put(190,34){\includegraphics*[width=4mm, height=2mm]{red}}

    \put(190,97){\includegraphics*[width=4mm, height=2mm]{blue}}
    \put(190,94){\includegraphics*[width=4mm, height=2mm]{red}}    

    \put(190,156){\includegraphics*[width=4mm, height=2mm]{blue}}
    \put(190,154){\includegraphics*[width=4mm, height=2mm]{red}}    




    \put(45,37){unpolarized}
    \put(45,34){$|m_{\chi_{b2}}|=2$}

    \put(45,97){unpolarized}
    \put(45,94){$|m_{\chi_{b2}}|=1$}

    \put(45,156){unpolarized}
    \put(45,153){$|m_{\chi_{b2}}|=0$}



    \put(120,37){unpolarized}
    \put(120,34){$|m_{\chi_{b2}}|=2$}

    \put(120,97){unpolarized}
    \put(120,94){$|m_{\chi_{b2}}|=1$}

    \put(120,136){unpolarized}
    \put(120,133){$|m_{\chi_{b2}}|=0$}

    
    \put(195,37){unpolarized}
    \put(195,34){$|m_{\chi_{b2}}|=2$}

    \put(195,97){unpolarized}
    \put(195,94){$|m_{\chi_{b2}}|=1$}    

    \put(195,156){unpolarized}
    \put(195,153){$|m_{\chi_{b2}}|=0$}        

  \end{picture}
  }
\caption {\small
  Angular distributions of simulated events in
  $\boldsymbol{\chi_{b2}(1P) \to \Y1S \gamma}$ decay.
  The blue curves corresponds to unpolarized events
  distribution and the red curves corresponds to specified polarized events
  distribution. All histograms are normalized by the corresponding integral. }
\label{fig:syst:polarization:angles_chib21p_ups1s}
\end{figure}
\begin{figure}[H]
  \setlength{\unitlength}{1mm}
  \centering
  \begin{picture}(150,180)
  	\put(0,0){
      \includegraphics*[width=75mm, height=45mm]{polarization/chib21p_ups1s_w2_ratio}
    }
    \put(0,60){
      \includegraphics*[width=75mm, height=45mm]{polarization/chib21p_ups1s_w0_ratio}
    }
    \put(75,60){
      \includegraphics*[width=75mm, height=45mm]{polarization/chib21p_ups1s_w1_ratio}
    }
    \put(0,120){
      \includegraphics*[width=75mm, height=45mm]{polarization/chib11p_ups1s_w0_ratio}
    }
    \put(75,120){
      \includegraphics*[width=75mm, height=45mm]{polarization/chib11p_ups1s_w1_ratio}
    }

    \put(55,0){$p_T^{\Y1S} \left[\gev\right]$}
    \put(55,60){$p_T^{\Y1S} \left[\gev\right]$}
    \put(55,120){$p_T^{\Y1S} \left[\gev\right]$}

    \put(130,60){$p_T^{\Y1S} \left[\gev\right]$}
    \put(130,120){$p_T^{\Y1S} \left[\gev\right]$}


    \put(0,20){\begin{sideways}$\eps_{unpol} / \eps_{m_{\chi_{b2}}}$\end{sideways}}
    \put(0,80){\begin{sideways}$\eps_{unpol} / \eps_{m_{\chi_{b2}}}$\end{sideways}}
    \put(0,140){\begin{sideways}$\eps_{unpol} / \eps_{m_{\chi_{b1}}}$\end{sideways}}

    \put(75,80){\begin{sideways}$\eps_{unpol} / \eps_{m_{\chi_{b2}}}$\end{sideways}}
    \put(75,140){\begin{sideways}$\eps_{unpol} / \eps_{m_{\chi_{b1}}}$\end{sideways}}

    % \put(65,37){$m_2$}
    % \put(65,97){$m_0$}
    % \put(140,97){$m_1$}

    % \put(65,157){$m_0$}
    % \put(140,157){$m_1$}

    % \put(25,37){\small $\chi_{\bm{b2}}(1P) \to \Y1S \gamma$}
    % \put(25,97){\small $\chi_{\bm{b2}}(1P) \to \Y1S \gamma$}
    % \put(25,157){\small $\chi_{\bm{b1}}(1P) \to \Y1S \gamma$}

    % \put(100,97){\small $\chi_{\bm{b2}}(1P) \to \Y1S \gamma$}
    % \put(100,157){\small $\chi_{\bm{b1}}(1P) \to \Y1S \gamma$}


    \put(25,37){\small $|m_{\chi_{b2}}|=2$}
    \put(25,97){\small $|m_{\chi_{b2}}|=0$}
    \put(25,157){\small $|m_{\chi_{b1}}|=0$}

    \put(100,97){\small $|m_{\chi_{b2}}|=1$}
    \put(100,157){\small $|m_{\chi_{b1}}|=1$}


    % \put(65,97){$m_0$}
    % \put(140,97){$m_1$}

    % \put(65,157){$m_0$}
    % \put(140,157){$m_1$}

  \end{picture}
\caption {\small
  Ratio between the efficiency for unpolarized events and the corresponding
  efficiency for polarized events in ${\chi_{b}(1P) \to \Y1S \gamma}$ decays.
  The results are shown in specified intervals of \Y1S transverse momentum.
}
\label{sec:syst:polarization:eratio_chib1p}
\end{figure}

\begin{figure}[H]
  \setlength{\unitlength}{1mm}
  \centering
  \begin{picture}(150,180)
    \put(0,0){
      \includegraphics*[width=75mm, height=45mm]{polarization/chib22p_ups1s_w2_ratio}
    }
    \put(0,60){
      \includegraphics*[width=75mm, height=45mm]{polarization/chib22p_ups1s_w0_ratio}
    }
    \put(75,60){
      \includegraphics*[width=75mm, height=45mm]{polarization/chib22p_ups1s_w1_ratio}
    }
    \put(0,120){
      \includegraphics*[width=75mm, height=45mm]{polarization/chib12p_ups1s_w0_ratio}
    }
    \put(75,120){
      \includegraphics*[width=75mm, height=45mm]{polarization/chib12p_ups1s_w1_ratio}
    }

    \put(55,0){$p_T^{\Y1S} \left[\gev\right]$}
    \put(55,60){$p_T^{\Y1S} \left[\gev\right]$}
    \put(55,120){$p_T^{\Y1S} \left[\gev\right]$}

    \put(130,60){$p_T^{\Y1S} \left[\gev\right]$}
    \put(130,120){$p_T^{\Y1S} \left[\gev\right]$}


    \put(0,20){\begin{sideways}$\eps_{unpol} / \eps_{m_{\chi_{b2}}}$\end{sideways}}
    \put(0,80){\begin{sideways}$\eps_{unpol} / \eps_{m_{\chi_{b2}}}$\end{sideways}}
    \put(0,140){\begin{sideways}$\eps_{unpol} / \eps_{m_{\chi_{b1}}}$\end{sideways}}

    \put(75,80){\begin{sideways}$\eps_{unpol} / \eps_{m_{\chi_{b2}}}$\end{sideways}}
    \put(75,140){\begin{sideways}$\eps_{unpol} / \eps_{m_{\chi_{b1}}}$\end{sideways}}


    \put(25,37){\small $|m_{\chi_{b2}}|=2$}
    \put(25,97){\small $|m_{\chi_{b2}}|=0$}
    \put(25,157){\small $|m_{\chi_{b1}}|=0$}

    \put(100,97){\small $|m_{\chi_{b2}}|=1$}
    \put(100,157){\small $|m_{\chi_{b1}}|=1$}


    % \put(65,97){$m_0$}
    % \put(140,97){$m_1$}

    % \put(65,157){$m_0$}
    % \put(140,157){$m_1$}

  \end{picture}
\caption {\small
  Ratio between the efficiency for unpolarized events and the corresponding
  efficiency for polarized events in ${\chi_{b}(2P) \to \Y1S \gamma}$ decays.
  The results are shown in specified intervals of \Y1S transverse momentum.
}
\label{sec:syst:polarization:eratio_chib2p}
\end{figure}






The systematic uncertainty for different polarization scenarios is estimated
as a maximum deviation of ratio between efficiency measured for unpolarized
particles and all possible polarization scenarios. The results  are
shown in~\Crefrange{tab:syst:pol:ups1s}{tab:syst:pol:ups3s}.

\begin{table}[H]
\caption{\small Maximum deviation (\%) of ratio between efficiency measured for unpolarized particles and all possible polarization scenarios in $\chi_{b} \to \Upsilon(1S) \gamma$ decays}
\centering
\scalebox{1}{
\begin{tabular}{lrrrrrr}\toprule
 & \multicolumn{6}{c}{$\Upsilon(1S)$ transverse momentum intervals, \gevc}\\
 & \multicolumn{1}{c}{6 -- 8} & \multicolumn{1}{c}{8 -- 10} & \multicolumn{1}{c}{10 -- 14} & \multicolumn{1}{c}{14 -- 18} & \multicolumn{1}{c}{18 -- 22} & \multicolumn{1}{c}{22 -- 40}\\
\midrule
$\chi_b(1P) \to \Y1S \gamma$ & ${}^{+2.4}_{-4.0}$ & ${}^{+3.5}_{-5.1}$ & ${}^{+2.9}_{-3.3}$ & ${}^{+1.1}_{-1.1}$ & ${}^{+2.3}_{-1.8}$ & ${}^{+4.0}_{-2.9}$\\

\rule{0pt}{4ex}$\chi_b(2P) \to \Y1S \gamma$ & ${}^{+0.9}_{-2.0}$ & ${}^{+0.9}_{-1.5}$ & ${}^{+0.7}_{-0.8}$ & ${}^{+2.7}_{-2.8}$ & ${}^{+5.3}_{-5.8}$ & ${}^{+6.8}_{-5.5}$\\

\rule{0pt}{4ex}$\chi_b(3P) \to \Y1S \gamma$ & --- & --- & ${}^{+2.2}_{-2.4}$ & ${}^{+5.2}_{-5.3}$ & ${}^{+6.7}_{-6.9}$ & ${}^{+5.9}_{-6.3}$\\
\bottomrule
\end{tabular}
} % scalebox
\label{tab:syst:pol:ups1s}
\end{table}

\begin{table}[H]
\caption{\small Maximum deviation (\%) of ratio between efficiency measured for unpolarized particles and all possible polarization scenarios in $\chi_{b} \to \Upsilon(2S) \gamma$ decays}
\centering
\scalebox{1}{
\begin{tabular}{lrrrr}\toprule
 & \multicolumn{4}{c}{$\Upsilon(2S)$ transverse momentum intervals, \gevc}\\
 & \multicolumn{1}{c}{18 -- 22} & \multicolumn{1}{c}{18 -- 24} & \multicolumn{1}{c}{22 -- 24} & \multicolumn{1}{c}{24 -- 40}\\
\midrule
$\chi_b(2P) \to \Y2S \gamma$ & ${}^{+7.8}_{-8.7}$ & --- & ${}^{+6.1}_{-3.6}$ & ${}^{+4.6}_{-4.3}$\\

\rule{0pt}{4ex}$\chi_b(3P) \to \Y2S \gamma$ & --- & ${}^{+2.7}_{-2.6}$ & --- & ${}^{+4.2}_{-4.5}$\\
\bottomrule
\end{tabular}
} % scalebox
\label{tab:syst:pol:ups2s}
\end{table}

\begin{table}[H]
\caption{\small Maximum deviation (\%) of ratio between efficiency measured for unpolarized particles and all possible polarization scenarios in $\chi_{b} \to \Upsilon(3S) \gamma$ decays}
\centering
\scalebox{1}{
\begin{tabular}{lr}\toprule
 & \multicolumn{1}{c}{$\Upsilon(3S)$ transverse momentum intervals, \gevc}\\
 & \multicolumn{1}{c}{27 -- 40}\\
\midrule
$\chi_b(3P) \to \Y3S \gamma$ & ${}^{+7.5}_{-6.4}$\\
\bottomrule
\end{tabular}
} % scalebox
\label{tab:syst:pol:ups3s}
\end{table}

\subsection{Summary of systematic uncertainties on the $\Upsilon$ fractions in
$\chi_b \to \Upsilon \gamma$ decays} The systematic uncertainties determined in
the previous subsections on the various term of~\Cref{eqn:master} give
corresponding uncertainties on the $\Upsilon$ fraction measured in this
studies. These uncertainties are recapped for each systematic source
in~\Cref{tab:syst:common,tab:syst:summary}. For simplicity,
\Cref{tab:syst:summary} shows only maximum deviation of systematic uncertainty
in all bins and energies for the corresponding decays.

\begin{table}[H]
\center
\caption{$\Upsilon$ fraction uncertainties common to all \chib decays (\%)}
\begin{tabular}{lc}
\toprule
$\Upsilon$ fit model & $\pm 0.7$ \\
$\gamma$ reconstruction & $\pm 3$ \\
\bottomrule
\end{tabular}
\label{tab:syst:common}
\end{table}


\begin{table}[H]
\center
\caption{Summary of $\Upsilon$ fraction systematic uncertainties (\%)}
% \resizebox{\textwidth}{!}{
\begin{tabular}{lcc}
\toprule
&  \chib fit model & \chib polarization\\
\midrule
\rule{0pt}{4ex}$\chib(1P) \to \Y1S \gamma$ & ${}^{+4.3}_{-5.8}$ & ${}^{+5.1}_{-4.0}$\\
\rule{0pt}{4ex}$\chib(2P) \to \Y1S \gamma$ & ${}^{+4.8}_{-6.2}$ & ${}^{+5.8}_{-6.8}$\\
\rule{0pt}{4ex}$\chib(3P) \to \Y1S \gamma$ & ${}^{+19.6}_{-16.6}$ & ${}^{+6.9}_{-6.7}$\\
\rule{0pt}{4ex}$\chib(2P) \to \Y2S \gamma$ & ${}^{+2.3}_{-7.0}$ & ${}^{+8.7}_{-7.8}$\\
\rule{0pt}{4ex}$\chib(3P) \to \Y2S \gamma$ & ${}^{+19.7}_{-19.9}$ & ${}^{+4.5}_{-4.2}$\\
\rule{0pt}{4ex}$\chib(3P) \to \Y3S \gamma$ & ${}^{+20.9}_{-27.6}$ & ${}^{+6.4}_{-7.5}$\\

\bottomrule
\end{tabular}
% }
\label{tab:syst:summary}
\end{table}

% \begin{table}[H]
\centering
\caption{\small \Y1S fraction systematic uncertainties  related to $\chi_b$ fit model.}
\subtable[$6 < p_T^{\Y1S} < 10 \gevc$] {
\scalebox{0.9}{
\begin{tabular}{lrrrr}\toprule
 & \multicolumn{4}{c}{$\Upsilon(1S)$ transverse momentum intervals, \gevc}\\
 & \multicolumn{2}{c}{6 -- 8} & \multicolumn{2}{c}{8 -- 10}\\
\cmidrule(r){2-3}\cmidrule(r){4-5}
 & \multicolumn{1}{c}{\sqs = 7\tev} & \multicolumn{1}{c}{\sqs = 8\tev} & \multicolumn{1}{c}{\sqs = 7\tev} & \multicolumn{1}{c}{\sqs = 8\tev}\\
\midrule
$\chi_b(1P) \to \Y1S \gamma$ & ${}^{+0.9 \%}_{-1.1 \%}$ & ${}^{+0.8 \%}_{-1.2 \%}$ & ${}^{+1.1 \%}_{-1.4 \%}$ & ${}^{+0.9 \%}_{-1.2 \%}$\\

\rule{0pt}{4ex}$\chi_b(2P) \to \Y1S \gamma$ & ${}^{+0.1 \%}_{-0.1 \%}$ & ${}^{+0.1 \%}_{-0.0 \%}$ & ${}^{+0.3 \%}_{-0.3 \%}$ & ${}^{+0.2 \%}_{-0.1 \%}$\\

\rule{0pt}{4ex}$\chi_b(3P) \to \Y1S \gamma$ & --- & --- & --- & ---\\
\bottomrule
\end{tabular}
} % scalebox

} % subtable
\subtable[$10 < p_T^{\Y1S} < 18 \gevc$] {
\scalebox{0.9}{
\begin{tabular}{lrrrr}\toprule
 & \multicolumn{4}{c}{$\Upsilon(1S)$ transverse momentum intervals, \gevc}\\
 & \multicolumn{2}{c}{10 -- 14} & \multicolumn{2}{c}{14 -- 18}\\
\cmidrule(r){2-3}\cmidrule(r){4-5}
 & \multicolumn{1}{c}{\sqs = 7\tev} & \multicolumn{1}{c}{\sqs = 8\tev} & \multicolumn{1}{c}{\sqs = 7\tev} & \multicolumn{1}{c}{\sqs = 8\tev}\\
\midrule
$\chi_b(1P) \to \Y1S \gamma$ & ${}^{+0.8 \%}_{-1.1 \%}$ & ${}^{+0.9 \%}_{-1.2 \%}$ & ${}^{+0.6 \%}_{-0.7 \%}$ & ${}^{+0.6 \%}_{-0.7 \%}$\\

\rule{0pt}{4ex}$\chi_b(2P) \to \Y1S \gamma$ & ${}^{+0.2 \%}_{-0.2 \%}$ & ${}^{+0.2 \%}_{-0.1 \%}$ & ${}^{+0.2 \%}_{-0.3 \%}$ & ${}^{+0.2 \%}_{-0.2 \%}$\\

\rule{0pt}{4ex}$\chi_b(3P) \to \Y1S \gamma$ & ${}^{+0.2 \%}_{-0.3 \%}$ & ${}^{+0.2 \%}_{-0.2 \%}$ & ${}^{+0.2 \%}_{-0.4 \%}$ & ${}^{+0.1 \%}_{-0.1 \%}$\\
\bottomrule
\end{tabular}
} % scalebox

} % subtable
\subtable[$18 < p_T^{\Y1S} < 40 \gevc$] {
\scalebox{0.9}{
\begin{tabular}{lrrrr}\toprule
 & \multicolumn{4}{c}{$\Upsilon(1S)$ transverse momentum intervals, \gevc}\\
 & \multicolumn{2}{c}{18 -- 22} & \multicolumn{2}{c}{22 -- 40}\\
\cmidrule(r){2-3}\cmidrule(r){4-5}
 & \multicolumn{1}{c}{\sqs = 7\tev} & \multicolumn{1}{c}{\sqs = 8\tev} & \multicolumn{1}{c}{\sqs = 7\tev} & \multicolumn{1}{c}{\sqs = 8\tev}\\
\midrule
$\chi_b(1P) \to \Y1S \gamma$ & ${}^{+0.5 \%}_{-0.9 \%}$ & ${}^{+0.5 \%}_{-0.9 \%}$ & ${}^{+0.6 \%}_{-0.8 \%}$ & ${}^{+0.9 \%}_{-1.6 \%}$\\

\rule{0pt}{4ex}$\chi_b(2P) \to \Y1S \gamma$ & ${}^{+0.2 \%}_{-0.2 \%}$ & ${}^{+0.2 \%}_{-0.2 \%}$ & ${}^{+0.2 \%}_{-0.5 \%}$ & ${}^{+0.4 \%}_{-0.1 \%}$\\

\rule{0pt}{4ex}$\chi_b(3P) \to \Y1S \gamma$ & ${}^{+0.4 \%}_{-0.3 \%}$ & ${}^{+0.1 \%}_{-0.1 \%}$ & ${}^{+0.1 \%}_{-0.3 \%}$ & ${}^{+0.4 \%}_{-0.2 \%}$\\
\bottomrule
\end{tabular}
} % scalebox

} % subtable
\label{tab:frac:ups1s_chib_model}
\end{table}

\begin{table}[H]
\centering
\caption{\small \Y1S fraction systematic uncertainties related to $\Upsilon$ fit model.}
\subtable[$6 < p_T^{\Y1S} < 10 \gevc$] {
\scalebox{0.9}{
\begin{tabular}{lrrrr}\toprule
 & \multicolumn{4}{c}{$\Upsilon(1S)$ transverse momentum intervals, \gevc}\\
 & \multicolumn{2}{c}{6 -- 8} & \multicolumn{2}{c}{8 -- 10}\\
\cmidrule(r){2-3}\cmidrule(r){4-5}
 & \multicolumn{1}{c}{\sqs = 7\tev} & \multicolumn{1}{c}{\sqs = 8\tev} & \multicolumn{1}{c}{\sqs = 7\tev} & \multicolumn{1}{c}{\sqs = 8\tev}\\
\midrule
$\chi_b(1P) \to \Y1S \gamma$ & $\pm 0.162 \%$ & $\pm 0.162 \%$ & $\pm 0.174 \%$ & $\pm 0.178 \%$\\

\rule{0pt}{4ex}$\chi_b(2P) \to \Y1S \gamma$ & $\pm 0.027 \%$ & $\pm 0.020 \%$ & $\pm 0.047 \%$ & $\pm 0.043 \%$\\

\rule{0pt}{4ex}$\chi_b(3P) \to \Y1S \gamma$ & --- & --- & --- & ---\\
\bottomrule
\end{tabular}
} % scalebox

} % subtable
\subtable[$10 < p_T^{\Y1S} < 18 \gevc$] {
\scalebox{0.9}{
\begin{tabular}{lrrrr}\toprule
 & \multicolumn{4}{c}{$\Upsilon(1S)$ transverse momentum intervals, \gevc}\\
 & \multicolumn{2}{c}{10 -- 14} & \multicolumn{2}{c}{14 -- 18}\\
\cmidrule(r){2-3}\cmidrule(r){4-5}
 & \multicolumn{1}{c}{\sqs = 7\tev} & \multicolumn{1}{c}{\sqs = 8\tev} & \multicolumn{1}{c}{\sqs = 7\tev} & \multicolumn{1}{c}{\sqs = 8\tev}\\
\midrule
$\chi_b(1P) \to \Y1S \gamma$ & $\pm 0.186 \%$ & $\pm 0.207 \%$ & $\pm 0.215 \%$ & $\pm 0.212 \%$\\

\rule{0pt}{4ex}$\chi_b(2P) \to \Y1S \gamma$ & $\pm 0.044 \%$ & $\pm 0.030 \%$ & $\pm 0.048 \%$ & $\pm 0.044 \%$\\

\rule{0pt}{4ex}$\chi_b(3P) \to \Y1S \gamma$ & $\pm 0.014 \%$ & $\pm 0.011 \%$ & $\pm 0.017 \%$ & $\pm 0.011 \%$\\
\bottomrule
\end{tabular}
} % scalebox

} % subtable
\subtable[$18 < p_T^{\Y1S} < 40 \gevc$] {
\scalebox{0.9}{
\begin{tabular}{lrrrr}\toprule
 & \multicolumn{4}{c}{$\Upsilon(1S)$ transverse momentum intervals, \gevc}\\
 & \multicolumn{2}{c}{18 -- 22} & \multicolumn{2}{c}{22 -- 40}\\
\cmidrule(r){2-3}\cmidrule(r){4-5}
 & \multicolumn{1}{c}{\sqs = 7\tev} & \multicolumn{1}{c}{\sqs = 8\tev} & \multicolumn{1}{c}{\sqs = 7\tev} & \multicolumn{1}{c}{\sqs = 8\tev}\\
\midrule
$\chi_b(1P) \to \Y1S \gamma$ & $\pm 0.234 \%$ & $\pm 0.219 \%$ & $\pm 0.244 \%$ & $\pm 0.236 \%$\\

\rule{0pt}{4ex}$\chi_b(2P) \to \Y1S \gamma$ & $\pm 0.047 \%$ & $\pm 0.034 \%$ & $\pm 0.052 \%$ & $\pm 0.061 \%$\\

\rule{0pt}{4ex}$\chi_b(3P) \to \Y1S \gamma$ & $\pm 0.014 \%$ & $\pm 0.009 \%$ & $\pm 0.025 \%$ & $\pm 0.017 \%$\\
\bottomrule
\end{tabular}
} % scalebox

} % subtable
\label{tab:frac:ups1s_ups_model}
\end{table}

\begin{table}[H]
\centering
\caption{\small \Y1S fraction systematic uncertainties related to photon reconstruction efficiency.}
\subtable[$6 < p_T^{\Y1S} < 10 \gevc$] {
\scalebox{0.9}{
\begin{tabular}{lrrrr}\toprule
 & \multicolumn{4}{c}{$\Upsilon(1S)$ transverse momentum intervals, \gevc}\\
 & \multicolumn{2}{c}{6 -- 8} & \multicolumn{2}{c}{8 -- 10}\\
\cmidrule(r){2-3}\cmidrule(r){4-5}
 & \multicolumn{1}{c}{\sqs = 7\tev} & \multicolumn{1}{c}{\sqs = 8\tev} & \multicolumn{1}{c}{\sqs = 7\tev} & \multicolumn{1}{c}{\sqs = 8\tev}\\
\midrule
$\chi_b(1P) \to \Y1S \gamma$ & $\pm 0.71 \%$ & $\pm 0.71 \%$ & $\pm 0.76 \%$ & $\pm 0.78 \%$\\

\rule{0pt}{4ex}$\chi_b(2P) \to \Y1S \gamma$ & $\pm 0.12 \%$ & $\pm 0.09 \%$ & $\pm 0.21 \%$ & $\pm 0.19 \%$\\

\rule{0pt}{4ex}$\chi_b(3P) \to \Y1S \gamma$ & --- & --- & --- & ---\\
\bottomrule
\end{tabular}
} % scalebox

} % subtable
\subtable[$10 < p_T^{\Y1S} < 18 \gevc$] {
\scalebox{0.9}{
\begin{tabular}{lrrrr}\toprule
 & \multicolumn{4}{c}{$\Upsilon(1S)$ transverse momentum intervals, \gevc}\\
 & \multicolumn{2}{c}{10 -- 14} & \multicolumn{2}{c}{14 -- 18}\\
\cmidrule(r){2-3}\cmidrule(r){4-5}
 & \multicolumn{1}{c}{\sqs = 7\tev} & \multicolumn{1}{c}{\sqs = 8\tev} & \multicolumn{1}{c}{\sqs = 7\tev} & \multicolumn{1}{c}{\sqs = 8\tev}\\
\midrule
$\chi_b(1P) \to \Y1S \gamma$ & $\pm 0.82 \%$ & $\pm 0.91 \%$ & $\pm 0.94 \%$ & $\pm 0.93 \%$\\

\rule{0pt}{4ex}$\chi_b(2P) \to \Y1S \gamma$ & $\pm 0.19 \%$ & $\pm 0.13 \%$ & $\pm 0.21 \%$ & $\pm 0.19 \%$\\

\rule{0pt}{4ex}$\chi_b(3P) \to \Y1S \gamma$ & $\pm 0.06 \%$ & $\pm 0.05 \%$ & $\pm 0.08 \%$ & $\pm 0.05 \%$\\
\bottomrule
\end{tabular}
} % scalebox

} % subtable
\subtable[$18 < p_T^{\Y1S} < 40 \gevc$] {
\scalebox{0.9}{
\begin{tabular}{lrrrr}\toprule
 & \multicolumn{4}{c}{$\Upsilon(1S)$ transverse momentum intervals, \gevc}\\
 & \multicolumn{2}{c}{18 -- 22} & \multicolumn{2}{c}{22 -- 40}\\
\cmidrule(r){2-3}\cmidrule(r){4-5}
 & \multicolumn{1}{c}{\sqs = 7\tev} & \multicolumn{1}{c}{\sqs = 8\tev} & \multicolumn{1}{c}{\sqs = 7\tev} & \multicolumn{1}{c}{\sqs = 8\tev}\\
\midrule
$\chi_b(1P) \to \Y1S \gamma$ & $\pm 1.03 \%$ & $\pm 0.96 \%$ & $\pm 1.07 \%$ & $\pm 1.04 \%$\\

\rule{0pt}{4ex}$\chi_b(2P) \to \Y1S \gamma$ & $\pm 0.20 \%$ & $\pm 0.15 \%$ & $\pm 0.23 \%$ & $\pm 0.27 \%$\\

\rule{0pt}{4ex}$\chi_b(3P) \to \Y1S \gamma$ & $\pm 0.06 \%$ & $\pm 0.04 \%$ & $\pm 0.11 \%$ & $\pm 0.07 \%$\\
\bottomrule
\end{tabular}
} % scalebox

} % subtable
\label{tab:frac:ups1s_eff}
\end{table}

\begin{table}[H]
\centering
\caption{\small \Y1S fraction systematic uncertainties related to unknown $\chi_b$ polarization.}
\subtable[$6 < p_T^{\Y1S} < 10 \gevc$] {
\scalebox{0.9}{
\begin{tabular}{lrrrr}\toprule
 & \multicolumn{4}{c}{$\Upsilon(1S)$ transverse momentum intervals, \gevc}\\
 & \multicolumn{2}{c}{6 -- 8} & \multicolumn{2}{c}{8 -- 10}\\
\cmidrule(r){2-3}\cmidrule(r){4-5}
 & \multicolumn{1}{c}{\sqs = 7\tev} & \multicolumn{1}{c}{\sqs = 8\tev} & \multicolumn{1}{c}{\sqs = 7\tev} & \multicolumn{1}{c}{\sqs = 8\tev}\\
\midrule
$\chi_b(1P) \to \Y1S \gamma$ & ${}^{+1.0 \%}_{-0.5 \%}$ & ${}^{+1.0 \%}_{-0.5 \%}$ & ${}^{+1.3 \%}_{-0.8 \%}$ & ${}^{+1.4 \%}_{-0.8 \%}$\\

\rule{0pt}{4ex}$\chi_b(2P) \to \Y1S \gamma$ & ${}^{+0.1 \%}_{-0.0 \%}$ & ${}^{+0.1 \%}_{-0.0 \%}$ & ${}^{+0.1 \%}_{-0.1 \%}$ & ${}^{+0.1 \%}_{-0.1 \%}$\\

\rule{0pt}{4ex}$\chi_b(3P) \to \Y1S \gamma$ & --- & --- & --- & ---\\
\bottomrule
\end{tabular}
} % scalebox

} % subtable
\subtable[$10 < p_T^{\Y1S} < 18 \gevc$] {
\scalebox{0.9}{
\begin{tabular}{lrrrr}\toprule
 & \multicolumn{4}{c}{$\Upsilon(1S)$ transverse momentum intervals, \gevc}\\
 & \multicolumn{2}{c}{10 -- 14} & \multicolumn{2}{c}{14 -- 18}\\
\cmidrule(r){2-3}\cmidrule(r){4-5}
 & \multicolumn{1}{c}{\sqs = 7\tev} & \multicolumn{1}{c}{\sqs = 8\tev} & \multicolumn{1}{c}{\sqs = 7\tev} & \multicolumn{1}{c}{\sqs = 8\tev}\\
\midrule
$\chi_b(1P) \to \Y1S \gamma$ & ${}^{+0.9 \%}_{-0.7 \%}$ & ${}^{+1.0 \%}_{-0.8 \%}$ & ${}^{+0.3 \%}_{-0.3 \%}$ & ${}^{+0.3 \%}_{-0.3 \%}$\\

\rule{0pt}{4ex}$\chi_b(2P) \to \Y1S \gamma$ & ${}^{+0.1 \%}_{-0.0 \%}$ & ${}^{+0.0 \%}_{-0.0 \%}$ & ${}^{+0.2 \%}_{-0.2 \%}$ & ${}^{+0.2 \%}_{-0.2 \%}$\\

\rule{0pt}{4ex}$\chi_b(3P) \to \Y1S \gamma$ & ${}^{+0.0 \%}_{-0.0 \%}$ & ${}^{+0.0 \%}_{-0.0 \%}$ & ${}^{+0.1 \%}_{-0.1 \%}$ & ${}^{+0.1 \%}_{-0.1 \%}$\\
\bottomrule
\end{tabular}
} % scalebox

} % subtable
\subtable[$18 < p_T^{\Y1S} < 40 \gevc$] {
\scalebox{0.9}{
\begin{tabular}{lrrrr}\toprule
 & \multicolumn{4}{c}{$\Upsilon(1S)$ transverse momentum intervals, \gevc}\\
 & \multicolumn{2}{c}{18 -- 22} & \multicolumn{2}{c}{22 -- 40}\\
\cmidrule(r){2-3}\cmidrule(r){4-5}
 & \multicolumn{1}{c}{\sqs = 7\tev} & \multicolumn{1}{c}{\sqs = 8\tev} & \multicolumn{1}{c}{\sqs = 7\tev} & \multicolumn{1}{c}{\sqs = 8\tev}\\
\midrule
$\chi_b(1P) \to \Y1S \gamma$ & ${}^{+0.6 \%}_{-0.7 \%}$ & ${}^{+0.6 \%}_{-0.7 \%}$ & ${}^{+1.0 \%}_{-1.3 \%}$ & ${}^{+1.0 \%}_{-1.3 \%}$\\

\rule{0pt}{4ex}$\chi_b(2P) \to \Y1S \gamma$ & ${}^{+0.4 \%}_{-0.3 \%}$ & ${}^{+0.3 \%}_{-0.2 \%}$ & ${}^{+0.4 \%}_{-0.5 \%}$ & ${}^{+0.5 \%}_{-0.6 \%}$\\

\rule{0pt}{4ex}$\chi_b(3P) \to \Y1S \gamma$ & ${}^{+0.2 \%}_{-0.1 \%}$ & ${}^{+0.1 \%}_{-0.1 \%}$ & ${}^{+0.2 \%}_{-0.2 \%}$ & ${}^{+0.2 \%}_{-0.1 \%}$\\
\bottomrule
\end{tabular}
} % scalebox

} % subtable
\label{tab:frac:ups1s_pol}
\end{table}

\begin{table}[H]
\centering
\caption{\small \Y2S fraction systematic uncertainties  related to $\chi_b$ fit model.}
\subtable[$18 < p_T^{\Y2S} < 24 \gevc$] {
\scalebox{0.9}{
\begin{tabular}{lrrrr}\toprule
 & \multicolumn{4}{c}{$\Upsilon(2S)$ transverse momentum intervals, \gevc}\\
 & \multicolumn{2}{c}{18 -- 22} & \multicolumn{2}{c}{18 -- 24}\\
\cmidrule(r){2-3}\cmidrule(r){4-5}
 & \multicolumn{1}{c}{\sqs = 7\tev} & \multicolumn{1}{c}{\sqs = 8\tev} & \multicolumn{1}{c}{\sqs = 7\tev} & \multicolumn{1}{c}{\sqs = 8\tev}\\
\midrule
$\chi_b(2P) \to \Y2S \gamma$ & ${}^{+0.5 \%}_{-1.8 \%}$ & ${}^{+0.1 \%}_{-1.2 \%}$ & --- & ---\\

\rule{0pt}{4ex}$\chi_b(3P) \to \Y2S \gamma$ & --- & --- & ${}^{+0.2 \%}_{-0.5 \%}$ & ${}^{+0.0 \%}_{-0.3 \%}$\\
\bottomrule
\end{tabular}
} % scalebox

} % subtable
\subtable[$22 < p_T^{\Y2S} < 40 \gevc$] {
\scalebox{0.9}{
\begin{tabular}{lrrrr}\toprule
 & \multicolumn{4}{c}{$\Upsilon(2S)$ transverse momentum intervals, \gevc}\\
 & \multicolumn{2}{c}{22 -- 24} & \multicolumn{2}{c}{24 -- 40}\\
\cmidrule(r){2-3}\cmidrule(r){4-5}
 & \multicolumn{1}{c}{\sqs = 7\tev} & \multicolumn{1}{c}{\sqs = 8\tev} & \multicolumn{1}{c}{\sqs = 7\tev} & \multicolumn{1}{c}{\sqs = 8\tev}\\
\midrule
$\chi_b(2P) \to \Y2S \gamma$ & ${}^{+0.8 \%}_{-2.4 \%}$ & ${}^{+0.6 \%}_{-1.9 \%}$ & ${}^{+0.0 \%}_{-1.2 \%}$ & ${}^{+0.2 \%}_{-1.0 \%}$\\

\rule{0pt}{4ex}$\chi_b(3P) \to \Y2S \gamma$ & --- & --- & ${}^{+0.0 \%}_{-0.9 \%}$ & ${}^{+0.6 \%}_{-0.6 \%}$\\
\bottomrule
\end{tabular}
} % scalebox

} % subtable
\label{tab:frac:ups2s_chib_model}
\end{table}

\begin{table}[H]
\centering
\caption{\small \Y2S fraction systematic uncertainties related to $\Upsilon$ fit model.}
\subtable[$18 < p_T^{\Y2S} < 24 \gevc$] {
\scalebox{0.9}{
\begin{tabular}{lrrrr}\toprule
 & \multicolumn{4}{c}{$\Upsilon(2S)$ transverse momentum intervals, \gevc}\\
 & \multicolumn{2}{c}{18 -- 22} & \multicolumn{2}{c}{18 -- 24}\\
\cmidrule(r){2-3}\cmidrule(r){4-5}
 & \multicolumn{1}{c}{\sqs = 7\tev} & \multicolumn{1}{c}{\sqs = 8\tev} & \multicolumn{1}{c}{\sqs = 7\tev} & \multicolumn{1}{c}{\sqs = 8\tev}\\
\midrule
$\chi_b(2P) \to \Y2S \gamma$ & $\pm 0.216 \%$ & $\pm 0.237 \%$ & --- & ---\\

\rule{0pt}{4ex}$\chi_b(3P) \to \Y2S \gamma$ & --- & --- & $\pm 0.024 \%$ & $\pm 0.023 \%$\\
\bottomrule
\end{tabular}
} % scalebox

} % subtable
\subtable[$22 < p_T^{\Y2S} < 40 \gevc$] {
\scalebox{0.9}{
\begin{tabular}{lrrrr}\toprule
 & \multicolumn{4}{c}{$\Upsilon(2S)$ transverse momentum intervals, \gevc}\\
 & \multicolumn{2}{c}{22 -- 24} & \multicolumn{2}{c}{24 -- 40}\\
\cmidrule(r){2-3}\cmidrule(r){4-5}
 & \multicolumn{1}{c}{\sqs = 7\tev} & \multicolumn{1}{c}{\sqs = 8\tev} & \multicolumn{1}{c}{\sqs = 7\tev} & \multicolumn{1}{c}{\sqs = 8\tev}\\
\midrule
$\chi_b(2P) \to \Y2S \gamma$ & $\pm 0.239 \%$ & $\pm 0.220 \%$ & $\pm 0.180 \%$ & $\pm 0.222 \%$\\

\rule{0pt}{4ex}$\chi_b(3P) \to \Y2S \gamma$ & --- & --- & $\pm 0.064 \%$ & $\pm 0.021 \%$\\
\bottomrule
\end{tabular}
} % scalebox

} % subtable
\label{tab:frac:ups2s_ups_model}
\end{table}

\begin{table}[H]
\centering
\caption{\small \Y2S fraction systematic uncertainties related to photon reconstruction efficiency.}
\subtable[$18 < p_T^{\Y2S} < 24 \gevc$] {
\scalebox{0.9}{
\begin{tabular}{lrrrr}\toprule
 & \multicolumn{4}{c}{$\Upsilon(2S)$ transverse momentum intervals, \gevc}\\
 & \multicolumn{2}{c}{18 -- 22} & \multicolumn{2}{c}{18 -- 24}\\
\cmidrule(r){2-3}\cmidrule(r){4-5}
 & \multicolumn{1}{c}{\sqs = 7\tev} & \multicolumn{1}{c}{\sqs = 8\tev} & \multicolumn{1}{c}{\sqs = 7\tev} & \multicolumn{1}{c}{\sqs = 8\tev}\\
\midrule
$\chi_b(2P) \to \Y2S \gamma$ & $\pm 0.95 \%$ & $\pm 1.04 \%$ & --- & ---\\

\rule{0pt}{4ex}$\chi_b(3P) \to \Y2S \gamma$ & --- & --- & $\pm 0.11 \%$ & $\pm 0.10 \%$\\
\bottomrule
\end{tabular}
} % scalebox

} % subtable
\subtable[$22 < p_T^{\Y2S} < 40 \gevc$] {
\scalebox{0.9}{
\begin{tabular}{lrrrr}\toprule
 & \multicolumn{4}{c}{$\Upsilon(2S)$ transverse momentum intervals, \gevc}\\
 & \multicolumn{2}{c}{22 -- 24} & \multicolumn{2}{c}{24 -- 40}\\
\cmidrule(r){2-3}\cmidrule(r){4-5}
 & \multicolumn{1}{c}{\sqs = 7\tev} & \multicolumn{1}{c}{\sqs = 8\tev} & \multicolumn{1}{c}{\sqs = 7\tev} & \multicolumn{1}{c}{\sqs = 8\tev}\\
\midrule
$\chi_b(2P) \to \Y2S \gamma$ & $\pm 1.05 \%$ & $\pm 0.96 \%$ & $\pm 0.79 \%$ & $\pm 0.97 \%$\\

\rule{0pt}{4ex}$\chi_b(3P) \to \Y2S \gamma$ & --- & --- & $\pm 0.28 \%$ & $\pm 0.09 \%$\\
\bottomrule
\end{tabular}
} % scalebox

} % subtable
\label{tab:frac:ups2s_eff}
\end{table}

\begin{table}[H]
\centering
\caption{\small \Y2S fraction systematic uncertainties related to unknown $\chi_b$ polarization.}
\subtable[$18 < p_T^{\Y2S} < 24 \gevc$] {
\scalebox{0.9}{
\begin{tabular}{lrrrr}\toprule
 & \multicolumn{4}{c}{$\Upsilon(2S)$ transverse momentum intervals, \gevc}\\
 & \multicolumn{2}{c}{18 -- 22} & \multicolumn{2}{c}{18 -- 24}\\
\cmidrule(r){2-3}\cmidrule(r){4-5}
 & \multicolumn{1}{c}{\sqs = 7\tev} & \multicolumn{1}{c}{\sqs = 8\tev} & \multicolumn{1}{c}{\sqs = 7\tev} & \multicolumn{1}{c}{\sqs = 8\tev}\\
\midrule
$\chi_b(2P) \to \Y2S \gamma$ & ${}^{+2.9 \%}_{-2.2 \%}$ & ${}^{+3.2 \%}_{-2.4 \%}$ & --- & ---\\

\rule{0pt}{4ex}$\chi_b(3P) \to \Y2S \gamma$ & --- & --- & ${}^{+0.1 \%}_{-0.1 \%}$ & ${}^{+0.1 \%}_{-0.1 \%}$\\
\bottomrule
\end{tabular}
} % scalebox

} % subtable
\subtable[$22 < p_T^{\Y2S} < 40 \gevc$] {
\scalebox{0.9}{
\begin{tabular}{lrrrr}\toprule
 & \multicolumn{4}{c}{$\Upsilon(2S)$ transverse momentum intervals, \gevc}\\
 & \multicolumn{2}{c}{22 -- 24} & \multicolumn{2}{c}{24 -- 40}\\
\cmidrule(r){2-3}\cmidrule(r){4-5}
 & \multicolumn{1}{c}{\sqs = 7\tev} & \multicolumn{1}{c}{\sqs = 8\tev} & \multicolumn{1}{c}{\sqs = 7\tev} & \multicolumn{1}{c}{\sqs = 8\tev}\\
\midrule
$\chi_b(2P) \to \Y2S \gamma$ & ${}^{+1.3 \%}_{-2.0 \%}$ & ${}^{+1.2 \%}_{-1.8 \%}$ & ${}^{+1.2 \%}_{-1.1 \%}$ & ${}^{+1.4 \%}_{-1.4 \%}$\\

\rule{0pt}{4ex}$\chi_b(3P) \to \Y2S \gamma$ & --- & --- & ${}^{+0.4 \%}_{-0.4 \%}$ & ${}^{+0.1 \%}_{-0.1 \%}$\\
\bottomrule
\end{tabular}
} % scalebox

} % subtable
\label{tab:frac:ups2s_pol}
\end{table}

\begin{table}[H]
\caption{\small \Y3S fraction systematic uncertainties  related to $\chi_b$ fit model.}
\centering
\scalebox{0.9}{
\begin{tabular}{lrr}\toprule
 & \multicolumn{2}{c}{$\Upsilon(3S)$ transverse momentum intervals, \gevc}\\
 & \multicolumn{2}{c}{27 -- 40}\\
\cmidrule(r){2-3}
 & \multicolumn{1}{c}{\sqs = 7\tev} & \multicolumn{1}{c}{\sqs = 8\tev}\\
\midrule
$\chi_b(3P) \to \Y3S \gamma$ & ${}^{+10.4 \%}_{-11.8 \%}$ & ${}^{+0.0 \%}_{-11.7 \%}$\\
\bottomrule
\end{tabular}
} % scalebox
\label{tab:frac:ups3s_chib_model}
\end{table}

\begin{table}[H]
\caption{\small \Y3S fraction systematic uncertainties related to $\Upsilon$ fit model.}
\centering
\scalebox{0.9}{
\begin{tabular}{lrr}\toprule
 & \multicolumn{2}{c}{$\Upsilon(3S)$ transverse momentum intervals, \gevc}\\
 & \multicolumn{2}{c}{27 -- 40}\\
\cmidrule(r){2-3}
 & \multicolumn{1}{c}{\sqs = 7\tev} & \multicolumn{1}{c}{\sqs = 8\tev}\\
\midrule
$\chi_b(3P) \to \Y3S \gamma$ & $\pm 0.286 \%$ & $\pm 0.299 \%$\\
\bottomrule
\end{tabular}
} % scalebox
\label{tab:frac:ups3s_ups_model}
\end{table}

\begin{table}[H]
\caption{\small \Y3S fraction systematic uncertainties related to photon reconstruction efficiency.}
\centering
\scalebox{0.9}{
\begin{tabular}{lrr}\toprule
 & \multicolumn{2}{c}{$\Upsilon(3S)$ transverse momentum intervals, \gevc}\\
 & \multicolumn{2}{c}{27 -- 40}\\
\cmidrule(r){2-3}
 & \multicolumn{1}{c}{\sqs = 7\tev} & \multicolumn{1}{c}{\sqs = 8\tev}\\
\midrule
$\chi_b(3P) \to \Y3S \gamma$ & $\pm 1.26 \%$ & $\pm 1.31 \%$\\
\bottomrule
\end{tabular}
} % scalebox
\label{tab:frac:ups3s_eff}
\end{table}

\begin{table}[H]
\caption{\small \Y3S fraction systematic uncertainties related to unknown $\chi_b$ polarization.}
\centering
\scalebox{0.9}{
\begin{tabular}{lrr}\toprule
 & \multicolumn{2}{c}{$\Upsilon(3S)$ transverse momentum intervals, \gevc}\\
 & \multicolumn{2}{c}{27 -- 40}\\
\cmidrule(r){2-3}
 & \multicolumn{1}{c}{\sqs = 7\tev} & \multicolumn{1}{c}{\sqs = 8\tev}\\
\midrule
$\chi_b(3P) \to \Y3S \gamma$ & ${}^{+2.8 \%}_{-2.8 \%}$ & ${}^{+2.9 \%}_{-3.0 \%}$\\
\bottomrule
\end{tabular}
} % scalebox
\label{tab:frac:ups3s_pol}
\end{table}




