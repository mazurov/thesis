\section{\texorpdfstring{$\Upsilon$}{Y} signal extraction}
\label{sec:upsilon}

%% ============================================================================
\subsection{Selection}
\label{sec:ups:selelection}
\subsubsection{Pre-selection}
\label{sec:upsilon:selelection:preselection}
The pre-selected event candidates were taken from datasets dedicated to
quarkonia studies in \lhcb. The selection starts by forming candidates from
pairs of oppositely-charged tracks identified as muons and originated from a
common vertex. Good track quality is ensured by requiring a $\chisq$ per number
of degree of freedom ($\chisq/\rm{ndf}$) to be less then 4 for the track fit
and primary vertex probability greater than 0.5 \%. The muons were required to
have a transverse momentum higher than 1 \gevc. To suppress duplicate tracks a
cut on the Kullback-Leibler~\cite{Needham:1082460} (KL) distance was used: only
tracks with symmetrized KL distance less than 5000 were selected\footnote{The
KL distance measures the difference between PDFs that describe track
parameters. If the distance is small then two tracks are likely to be
clones.}.The primary vertex of dimuon candidate is required to be within
luminous region defined as $|z_{PV}| < 0.5 m$ and $x_{PV}^2 + y_{PV}^2 < 100
mm^2$.
%============================================================================
\subsubsection{Trigger}
\label{sec:upsilon:selection:trigger}

For Upsilon studies the event candidates  were required to trigger by three
trigger levels ('TOS' requirement). The first level is L0DiMuon which  requires
two muon candidates with their $p_T$ product larger than 1.68$\gev^2/c^2$ and
a relaxed SPD requirement (less than 9000 SPD hits).

The second level is HLT1 trigger where the event candidates were required to
pass the Hlt1DiMuonHighMass line.
This line triggers events with  two well reconstructed tracks  which have hits
in the muon system  that have a transverse momentum higher that 500\mevc and 
a total momentum higher than 6\gevc that produced from a common vertex with
an invariant mass higher than 2.7\gevcc.

At the last HLT2 level the event needs to be accepted by the HLT2DiMuonB line.
This line confirms the HLT1 decision and requires the invariant mass of the
dimuon pair to be larger than 4.7\gevcc.

%% ============================================================================
\subsubsection{This study selection}
\label{sec:upsilon:selection:study}

To improve the muon identification purity two additional cuts are used. The
first one is applied on the difference in logarithm of the likelihood of the
muon and hadron hypotheses~\cite{Powell} provided by the muon detection system.
This difference ($\Delta\log\lum^{\mu-\Ph}$) should be greater than 0. The
second selection  is a cut on the muon probability value obtained from a Neural
Network algorithm (ProbNN). This algorithm takes in account the various
reconstruction information such as the RICH particle identification, the muon
quality reconstruction and compatibility with a minimum ionising particle in the
calorimeters. In this study selected events requiring to have the ProbNN value
greater than $0.5$.

The criteria for  $\Upsilon$ selection are summarized in
Table~\ref{tab:upsilon:selection:study:summary}.

\begin{table}[H]
\caption{\small Summary of $\Upsilon$ selection criteria}
\centering
\begin{tabular}{cl}\toprule
Description & Requirement \\
\midrule
$\Upsilon$ rapidity range &  $2.0 < y^{\Upsilon} < 4.5$ \\
Track fit quality & $\chisq/\rm{ndf} < 4$ \\
Track transverse momentum & $> 1$ \gevc \\
$\mumu$ transverse momentum & $6 < \pt(\mumu) < 40 \gevc$ \\
Primary vertex probability & $> 0.5 \%$ \\
Luminous region & $|z_{PV}| < 0.5 m$ and $x_{PV}^2 + y_{PV}^2 < 100 mm^2$ \\
Kullback-Leibler distance & $> 5000$ \\
\rule{0pt}{4ex}Muon and hadron hypotheses & $\Delta\log\lum^{\mu-\Ph} > 0$ \\
Muon probability & ProbNN $> 0.5$ \\
\multicolumn{2}{l}{\rule{0pt}{4ex}Trigger lines:} \\
L0 & L0DiMuon \\
HLT1 & Hlt1DiMuonHighMass \\
HLT2 & HLT2DiMuonB \\
\bottomrule
\end{tabular}
\label{tab:upsilon:selection:study:summary}
\end{table}

%% ============================================================================
\subsection{Fit model}
\label{sec:upsilon:fit}
To determine the $\Upsilon$ meson signal yields, an unbinned maximum likelihood
fit to dimuon mass distribution has been performed. The signals have been
modeled with the sum of three double-sided CrystalBall (DSCB) functions and the
combinatorial background by an exponential function with floating $\tau$
parameter. Each of the DSCB functions corresponds to \Y1S, \Y2S and \Y3S
signals and can be written in the following form:

\begin{equation}
DSCB(x) = N \times
\begin{cases}
\frac{1}{\sqrt{2\pi\sigma}}{(\frac{n_L}{|\alpha_L|})}^{n_L}\exp(-\frac{|\alpha_L|^2}{2}){(\frac{n_L}{|\alpha_L|}-|\alpha_L|-\frac{x-\mu}{\sigma})}^{-n_L} & \text{, if $\frac{x-\mu}{\sigma} < -\alpha_L$}\\
\frac{1}{\sqrt{2\pi\sigma}}{(\frac{n_R}{|\alpha_R|})}^{n_R}\exp(-\frac{|\alpha_R|^2}{2}){(\frac{n_R}{|\alpha_R|}-|\alpha_R|+\frac{x-\mu}{\sigma})}^{-n_R} & \text{, if $\frac{x-\mu}{\sigma} > \alpha_R$}\\
\frac{1}{\sqrt{2\pi\sigma}}\exp(-\frac{{(x-\mu)}^2}{2\sigma^2}) & \text{, otherwise}
\end{cases}
\label{eq:dcb}
\end{equation}

The double-sided CrystalBall is similar to gaussian
distribution, but has an asymmetric tails. This function has seven parameters:
N, $\mu$, $\sigma$, $\alpha_L$, $n_L$, $\alpha_R$, $n_R$ where parameters $\mu$
and $\sigma$ have the same meaning as for gaussian. Parameters $\alpha_L
(\alpha_R)$ and $n_L (n_R)$ describe the left (right) tail behavior:
$\alpha_{L,R}$ controls the tail start and $n_{L,R}$ corresponds to the
decreasing power of the tail. All five parameters $N$, $\alpha_L$, $n_L$,
$\alpha_R$ and $n_R$ contribute to the amplitude.

% The mass of \Y1S is allowed to vary. Mass of \Y2S (\Y3S) are defined as sum of
% \Y1S mass and the value of corresponding difference between mass of
% \Y2S (\Y3S) and \Y1S. These two mass differences, named $\Delta m_{\Y2S}$ and
% $\Delta m_{\Y3S}$, are fixed in the fit to PDG~\cite{PDG2012} values 563 \mevcc
% and 895 \mevcc correspondingly.

% The width of the Crystal Ball function describing the \Y1S meson is allowed to
% vary, while the width of \Y2S and \Y3S mesons are constrained to the value of
% the width of the \Y1S signal, scaled by the ratio of the masses of \Y2S and \Y3S
% to the \Y1S meson.

The  $\alpha_{L,R}$ and $n_{L,R}$ parameters are fixed to the values extracted
from fits to the simulated $\Upsilon \to \mumu$ decays. The $\alpha_L$ and
$\alpha_R$ values are fixed to 1.6. The values of $n_L$  and $n_R$ are fixed
correspondingly to 2 and 11 in each of three double-sided CrystalBall
functions. All other parameters are allowed to vary in the fit model.

%% ============================================================================
\subsection{Fit results}
\label{sec:upsilon:result}

Figure~\ref{fig:upsilon:result:nominal} presents the result of the fit that is
described in the previous section. Fit was performed in the dimuon
transverse momentum interval $ 6 < p_T^{\mumu} < 40 \gevc$.
Table~\ref{tab:upsilon:result:nominal} shows the parameters values of this fit.

\begin{figure}[H]
  \setlength{\unitlength}{1mm}
  \centering
  \begin{picture}(150,60)
    \put(0,0){
      \includegraphics*[width=75mm, height=60mm]{upsilon/f2011_6_40}
    }
    \put(0,15){\small \begin{sideways}Candidates/(6\mevcc)\end{sideways}}
    \put(25, 2){$m_{\mumu} \left[\gevcc\right]$}
    \put(45,45){\sqs = 7 \tev}

    \put(75,0){
      \includegraphics*[width=75mm, height=60mm]{upsilon/f2012_6_40}
    }
    \put(75,15){\small \begin{sideways}Candidates/(6\mevcc)\end{sideways}}
    \put(100,2){$m_{\mumu} \left[\gevcc\right]$}
    \put(120,45){\sqs = 8 \tev}

    % \graphpaper[5](0,0)(150, 60)
  \end{picture}
  \caption {\small
    Invariant mass distibution of the selected $\Upsilon \to \mumu$ candidates
    in the range $ 6 < p_{T}^{\mumu}  < 40 \gevc$ and $2 < y^{\mumu} < 4.5 $.
    Three peaks correspond to the \Y1S, \Y2S and \Y2S signals (from left to
    right). Curves are the result of the fit described in the previous
    section~\ref{sec:upsilon:fit}. }
  \label{fig:upsilon:result:nominal}
\end{figure}
\begin{table}[H]
\caption{\small \mumu invariant mass data fit parameters}
\centering
\resizebox{.75\textwidth}{!}{
\begin{tabular}{lrr}\toprule
 & \multicolumn{2}{c}{$\mumu$ transverse momentum intervals, \gevc}\\
 & \multicolumn{2}{c}{6 -- 40}\\
\cmidrule(r){2-3}
 & \multicolumn{1}{c}{\sqs = 7\tev} & \multicolumn{1}{c}{\sqs = 8\tev}\\
\midrule
$N_{\Y1S}$ & 283,300 $\pm$ 600 & 659,600 $\pm$ 900\\
$N_{\Y2S}$ & 87,500 $\pm$ 400 & 203,300 $\pm$ 600\\
$N_{\Y3S}$ & 50,420 $\pm$ 290 & 115,300 $\pm$ 400\\

\rule{0pt}{4ex}Background & 296,400 $\pm$ 700 & 721,300 $\pm$ 1100\\

\rule{0pt}{4ex}$\mu_{\Y1S}, \mevcc$ & 9457.02 $\pm$ 0.10 & 9455.58 $\pm$ 0.07\\
$\sigma_{\Y1S}, \mevcc$ & 42.86 $\pm$ 0.10 & 43.04 $\pm$ 0.06\\

\rule{0pt}{4ex}$\mu_{\Y2S}, \mevcc$ & 10,019.03 $\pm$ 0.21 & 10,018.05 $\pm$ 0.14\\
$\sigma_{\Y2S}, \mevcc$ & 46.38 $\pm$ 0.20 & 46.45 $\pm$ 0.14\\

\rule{0pt}{4ex}$\mu_{\Y3S}, \mevcc$ & 10,351.16 $\pm$ 0.32 & 10,349.41 $\pm$ 0.16\\
$\sigma_{\Y3S}, \mevcc$ & 48.63 $\pm$ 0.31 & 48.24 $\pm$ 0.11\\

$\tau$ & -0.3887 $\pm$ 0.0023 & -0.3819 $\pm$ 0.0015\\
\bottomrule
\end{tabular}
} % scalebox
\label{tab:upsilon:result:nominal}
\end{table}

\begin{figure}[H]
  \setlength{\unitlength}{1mm}
  \centering
  \begin{picture}(150,60)
    \put(0,0){
      \includegraphics*[width=75mm, height=60mm]{upsilon/m1s}
    }
    \put(75,0){
      \includegraphics*[width=75mm, height=60mm]{upsilon/sigma}
    }

    \put(35,2){\small $p_T^{\mumu} \left[\gevc\right]$ }
    \put(110,2){\small $p_T^{\mumu} \left[\gevc\right]$ }

    \put(0,15){\small \begin{sideways}\Y1S mass $\left[\gevcc\right]$\end{sideways}}
    \put(75,10){\small \begin{sideways}\Y1S  resolution $\left[\gevcc\right]$\end{sideways}}

    \put(15,50){(a)}
    \put(90,50){(b)}

    \put(50,50){\textcolor{blue}{\sqs=7\tev}}
    \put(50,45){\textcolor{red}{\sqs=8\tev}}
    \put(44,50){
      \includegraphics*[width=4mm, height=2mm]{blue}
    }
    \put(44,45){
      \includegraphics*[width=4mm, height=2mm]{red}
    }

    \put(125,30){\textcolor{blue}{\sqs=7\tev}}
    \put(125,25){\textcolor{red}{\sqs=8\tev}}
    \put(119,30){
      \includegraphics*[width=4mm, height=2mm]{blue}
    }
    \put(119,25){
      \includegraphics*[width=4mm, height=2mm]{red}
    }    


  % \graphpaper[5](0,0)(75, 60)
  \end{picture}
  \caption {\small
    Distribution of \Y1S mass (a) and peak resolution (b) in $\Y1S \to \mumu$ decay as function of \mumu transverse momentum.
  }
  \label{fig:upsilon:result:mean_sigma}
\end{figure}

% \Cref{fig:upsilon:result:mean_sigma}(a) shows the fitted \Y1S mass is about
% 5 \mevcc lower than PDG value $9460.30 \pm  0.26$ \mevcc.

% \input{upsilon/result/pics/mass}


To obtain the final numbers for $\Upsilon$ yields the fit was repeated
independently for each of the $p_T^{\mumu}$ bins with the \OneS mass fixed to
$9.456 \gevcc$ that was measured in the fit of joined 2011 and 2012 datasets.
\Cref{fig:upsilon:result:mean_sigma} shows  the fitted $\Upsilon(1S)$
mass differs by 5 \mevcc from the PDG value $9460.30 \pm  0.26$ \mevcc and
varies as a function of transverse momentum, as also observed in other
studies~\cite{Aaij:2013yaa}. A systematic uncertainty is assigned due to this
effect.

Figure~\ref{fig:upsilon:result:yields} shows the number of signal events as
function of dimuon transverse momentum. Table~\ref{tab:upsilon:result:fits} in
Appendix summarizes the obtained results. Figure
\ref{fig:upsilon:result:yields_scaled} shows the $\Upsilon(nS)$ yields as a
function of transverse momentum, normalized by bin size and luminosity. The
small difference between 7 and 8 \tev data is due to the production cross
section, which is expected to rise by about 10\% for the latter case.


\begin{figure}[H]
  \setlength{\unitlength}{1mm}
  \centering
  \begin{picture}(150,120)
    \put(0,0){
      \includegraphics*[width=75mm, height=60mm]{upsilon/N3S}
    }
    \put(0,60){
      \includegraphics*[width=75mm, height=60mm]{upsilon/N1S}
    }
    \put(75,60){
      \includegraphics*[width=75mm, height=60mm]{upsilon/N2S}
    }

    \put(0,25){\begin{sideways}Events\end{sideways}}
    \put(35,2){$p_T(\mumu) \left[\gevc\right]$}
    \put(55,50){$\Y3S$}

    \put(0,85){\begin{sideways}Events\end{sideways}}
    \put(35,62){$p_T(\mumu) \left[\gevc\right]$}
    \put(55,110){$\Y1S$}

    \put(75,85){\begin{sideways}Events\end{sideways}}
    \put(110,62){$p_T(\mumu) \left[\gevc\right]$}
    \put(130,110){$\Y2S$}


    \put(50,45){\textcolor{blue}{\sqs=7\tev}}
    \put(50,40){\textcolor{red}{\sqs=8\tev}}
    \put(44,45){
      \includegraphics*[width=4mm, height=2mm]{blue}
    }
    \put(44,40){
      \includegraphics*[width=4mm, height=2mm]{red}
    }

    \put(50,105){\textcolor{blue}{\sqs=7\tev}}
    \put(50,100){\textcolor{red}{\sqs=8\tev}}
    \put(44,105){
      \includegraphics*[width=4mm, height=2mm]{blue}
    }
    \put(44,100){
      \includegraphics*[width=4mm, height=2mm]{red}
    }

    \put(125,105){\textcolor{blue}{\sqs=7\tev}}
    \put(125,100){\textcolor{red}{\sqs=8\tev}}
    \put(119,105){
      \includegraphics*[width=4mm, height=2mm]{blue}
    }
    \put(119,100){
      \includegraphics*[width=4mm, height=2mm]{red}
    }


  % \graphpaper[5](0,0)(75, 60)
  \end{picture}
  \caption {\small
    Distribution of $\Upsilon$ raw yields in $\Upsilon \to \mumu$ decay as function of transverse momentum.
  }
  \label{fig:upsilon:result:yields}
\end{figure}

\begin{figure}[H]
  \setlength{\unitlength}{1mm}
  \centering
  \begin{picture}(150,120)
    \put(0,0){
      \includegraphics*[width=75mm, height=60mm]{upsilon/N3S_scaledbylum}
    }
    \put(0,60){
      \includegraphics*[width=75mm, height=60mm]{upsilon/N1S_scaledbylum}
    }
    \put(75,60){
      \includegraphics*[width=75mm, height=60mm]{upsilon/N2S_scaledbylum}
    }

    \put(0,25){\begin{sideways}Arbitrary units\end{sideways}}
    \put(35,2){$p_T(\mumu) \left[\gevc\right]$}
    \put(55,50){$\Y3S$}

    \put(0,85){\begin{sideways}Arbitrary units\end{sideways}}
    \put(35,62){$p_T(\mumu) \left[\gevc\right]$}
    \put(55,110){$\Y1S$}

    \put(75,85){\begin{sideways}Arbitrary units\end{sideways}}
    \put(110,62){$p_T(\mumu) \left[\gevc\right]$}
    \put(130,110){$\Y2S$}


    \put(50,45){\textcolor{blue}{\sqs=7\tev}}
    \put(50,40){\textcolor{red}{\sqs=8\tev}}
    \put(44,45){
      \includegraphics*[width=4mm, height=2mm]{blue}
    }
    \put(44,40){
      \includegraphics*[width=4mm, height=2mm]{red}
    }

    \put(50,105){\textcolor{blue}{\sqs=7\tev}}
    \put(50,100){\textcolor{red}{\sqs=8\tev}}
    \put(44,105){
      \includegraphics*[width=4mm, height=2mm]{blue}
    }
    \put(44,100){
      \includegraphics*[width=4mm, height=2mm]{red}
    }

    \put(125,105){\textcolor{blue}{\sqs=7\tev}}
    \put(125,100){\textcolor{red}{\sqs=8\tev}}
    \put(119,105){
      \includegraphics*[width=4mm, height=2mm]{blue}
    }
    \put(119,100){
      \includegraphics*[width=4mm, height=2mm]{red}
    }


  % \graphpaper[5](0,0)(75, 60)
  \end{picture}
  \caption {\small
    Distribution of $\Upsilon$ yields in $\Upsilon \to \mumu$ decay in
    specified $\mumu$ transverse momentum ranges. The distributions are
    obtained by normalizing the raw yields by bin size and luminosity value. }
  \label{fig:upsilon:result:yields_scaled}
\end{figure}

